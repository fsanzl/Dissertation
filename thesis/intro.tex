\chapter{Introducción}
\epigraphhead[50]{\epigraphtextposition{flushleftright}\begin{german}
		\epigraph{Er zweifelte nun nicht, dass diese Gedichte ihm seinen Weg zum Theater noch mehr bahnen, und ihn bald seinem Ziele näher bringen würden.}{Karl Philip Moritz, \textit{Anton Reiser}}
\end{german}
}
Hemos de morir. Esta obviedad tiene implicaciones profundas no solo en nuestra forma de entender el mundo y a nosotros mismos, sino en cómo y hasta dónde podemos enfrentarnos a los problemas que nos encontramos durante la vida. Poco importan nuestras intenciones, creencias o deseos cuando llega el momento postrero: somos impotentes ante esa inapelable realidad. Esto hace que las aproximaciones a empresas de cierta envergadura hayan de ceñirse a un improrrogable límite temporal. Cuán factible es llevarlas a término, en muchos casos, depende de comprender la naturaleza perecedera de la existencia y adaptar nuestro propósito a la inevitabilidad del destino. Con frecuencia, no queda otra elección que resignarse a una solución de compromiso. 

Los estudios filológicos no son una excepción a esta regla universal, como ya notó \citeauthor{bloom1995}~\parencite*[29]{bloom1995}. El crítico apelaba precisamente la mortalidad del hombre para defender el canon occidental, arguyendo que \blockquote{\begin{english}there is only so much time, and time must have a stop, while there is more to read than there ever was before\end{english}}. En efecto, no podemos dedicar una lectura minuciosa a cada una de las obras que componen un corpus de cierta extensión, por lo que debemos conformarnos con intentar aprehender la esencia del conjunto en una reducidísima fracción de sus elementos\footnote{De ahí la necesidad de seleccionar con cierto criterio, \blockquote{\begin{english}we possess the Canon because we are mortal and also rather belated\end{english}} \parencite[29]{bloom1995}.}. 

Podemos tomar la literatura universal, o la nacional de un país, un género, un período, una lengua, incluso un tema dado en un periodo concreto y en un solo lugar: a mayor o menor escala, volvemos a toparnos con el mismo obstáculo, como en un gráfico fractal en el que un patrón reaparece recursivamente hasta el infinito cada vez que nos aproximamos. Igual ocurre con el teatro del Siglo de Oro, para el que los recursos disponibles difícilmente alcanzan a cubrir siquiera una esquirla de lo escrito en aquella época tan fructífera; considerar todos y cada uno de los versos de centenares de obras se antoja una tarea como la de aquel párvulo que intentaba vaciar el Mediterráneo con su pequeña concha ante el desconcertado sabio de Hipona.

\section{Objetivos}
El sonido y el ritmo de las palabras arrastran carga semántica en no pocas ocasiones, sea o no de manera intencionada. Esto es particularmente cierto en las obras teatrales del Siglo de Oro, en las que determinado tipo de verso se presta más fácilmente a ser empleado en determinadas coyunturas. Los dramaturgos de la época no dejaron de señalarlo, como no podía ser de otra forma, y el paradigma lo encontramos escrito de la pluma del mismo Lope de Vega \parencite*[305-312]{vega2006}: 
\blockquote{Acomode los versos con prudencia\\
	a los sujetos que va tratando.\\
	Las décimas son buenas para las quejas;\\
	el soneto está bien en los que aguardas;\\
	las relaciones piden los romances,\\
	aunque en octavas lucen por extremo;\\
	son los tercetos para cosas graves,\\
	y para las de amor, las redondillas.}
No ha de llamar a asombro que la crítica contemporánea también haya reparado en ello, sin olvidar las menciones recientes a la relación que parecen mostrar circunstancias y personajes de tipos muy concretos con sonidos \parencites{kroll2020a,kroll2019a}[36]{dominguez2005} y cadencias \parencite{sanchez2017} particulares, si bien la comedia áurea no tendría un simbolismo tan definido como el de las antiguas representaciones rituales \parencite[122]{aichinger2012}. Los grandes dramaturgos auriseculares acomodan el metro a la fábula con un cuidado exquisito para propiciar un acompañamiento sonoro adecuado a la acción \parencite[48-60]{aichinger2015a}. Sin embargo, establecer una correlación a ciencia cierta requiere analizar caso por caso un considerable número de situaciones dramáticas, lo que implica, asimismo, la obligación de escandir un sinnúmero de versos, de achicar el mar con una minúscula concha.

Por fortuna, la tecnología ayuda a contrarrestar estas carencias, si bien dentro de sus posibilidades. Aunque el computador\footnote{Hay división de opiniones en cuanto al término más adecuado para designar la máquina que en inglés se bautizó como \textit{computer}. Mientras el español americano opta por la versión anglosajona, de forma indistinta en su variante masculina \textit{computador} o femenina \textit{computadora}, la variedad peninsular, a semblanza del \textit{ordinateur} del vecino francés, se inclina por \textit{ordenador}. Antonio Vaquero Sánchez, pionero de la computación en España y coautor de uno de los más rigurosos glosarios de terminología informática publicados \parencite{vaquero1985} en español, entiende que ordenar es solo una pequeña parte de los problemas susceptibles de ser computables, por lo que considera inadecuado el calco del francés. Atendiendo a las razones que expone \parencite[9-11]{vaquero1999}, optaremos aquí por el término \textit{computador}.} se limita a ejecutar una serie de instrucciones relativamente sencillas\footnote{La eficacia de la electrónica depende al final del buen hacer del encargado de dictar dichas instrucciones.}, lo hace a una velocidad prodigiosa y sin acusar el cansancio que, tarde o temprano, termina por abatir al más afanoso de los humanos. Esta extensión técnica de nuestras capacidades la han venido usando la lingüística computacional y la de corpus con resultados excepcionales, pero también ha llegado a ese otro lugar difuso que se ha dado en llamar humanidades digitales\index{humanidades digitales}.

No aspiramos a demostrar aquí la —por otra parte, evidente— genialidad de Calderón para evocar emociones mediante la escenificación de pasiones ocultas \parencite[451]{bidwell2013} ni tampoco a explicar su singularidad con un estudio cuantitativo. Sí pretendemos, en cambio, ser capaces de extraer datos que reflejen categorías textuales y que estos posibiliten analizar diferencias cuantitativas. En último término, deseamos emplear herramientas prestadas de la lingüística en la creación de otras nuevas que faciliten al crítico —\nolinebreak al humano\nolinebreak— la identificación e interpretación de esas divergencias. Es el filólogo, conforme a sus conocimientos de la disciplina, quien decide qué observar, la forma de hacerlo y cómo entenderlo, pero delega en la máquina la extracción y clasificación de los datos. Nuestra labor es, pues, «enseñarle» al computador qué debe encontrar, cómo ha de llevarlo a cabo y de qué manera tiene que presentar los resultados. La disponibilidad de un enorme volumen de datos cuantificables obtenidos así da pie a descubrir patrones. A grandes rasgos, esto es lo que se ha dado en llamar \textit{lectura distante}\index{lectura distante}\index{distant reading@{\textit{distant reading}}|see {lectura distante}} \parencite{moretti2000}, sobre la que hablaremos más en detalle en el siguiente capítulo.

La lectura distante\index{lectura distante}, empero, demanda datos estructurados, textos aptos para el tratamiento digital. Asimismo, estos deben obedecer a criterios filológicos pertinentes: si son inapropiados, invalidarían los resultados de cualquier análisis, por impecable que fuera la metodología empleada. Por lo tanto, la lectura distante requiere, en primer lugar, un corpus\index{corpus} con la calidad óptima requerida por la investigación \parencite{calvo2019} y, en segundo, que este se halle dispuesto para extraer información precisa, veraz y relevante.

En el caso del drama español clásico, nos enfrentamos a un doble reto: por un lado, necesitamos textos anotados\index{anotar}\footnote{Aquí y en adelante uso el calco del inglés \textit{anotar} —\nolinebreak de cierto predicamento en el ámbito de la lingüística\nolinebreak— para referirme, en general, a enriquecer un texto con metadatos, mientras que relego el tal vez más adecuado término \textit{etiquetar} para el sentido más concreto de marcar con una etiqueta un segmento textual.} en virtud de sus características dramáticas. Por otro, al estar los parlamentos versificados, debería trabajarse con datos que tengan también en cuenta las particularidades líricas de los textos. Por lo tanto, de embarcarnos en el análisis filológico digital del verso teatral, deberíamos satisfacer primero esos requisitos antes de emprender la labor.

Esta tesis doctoral surge en el marco del proyecto Sound and Meaning in Spanish Golden Age Literature\footnote{Sound and Meaning en adelante.}, encabezado por Simon Kroll, como respuesta a la doble necesidad planteada arriba.  Entre las muchas ocupaciones del proyecto, se encuentra la de llevar a cabo diversos tipos de análisis de textos teatrales. Particularmente, trata de hallar relaciones entre las características rítmicas de los versos y varios de los aspectos estructurales y semánticos de las obras \parencite{kroll2022a}. Para poder comparar grandes cantidades de versos teniendo en cuenta su función en la estructura dramática, se requería un corpus de piezas del género apto para el análisis digital de esas propiedades. En esta tesis abordamos la cuestión de cómo obtenerlo. Lo hacemos mediante el desarrollo teórico y analítico de los principios rectores de métodos y herramientas \parencite{sanz2023b} que hemos concebimos para ello. Alcanzaremos esa meta tomando prestadas técnicas de la lingüística computacional y de corpus, que trasladaremos a los estudios teatrales áureos. Este trabajo se ocupa del examen obras dadas para extraer y organizar sus datos, mientras que son responsabilidad del proyecto la selección de los componentes del corpus y los experimentos que se valen de él.

Adicionalmente, lo aquí propuesto encuentra utilidad en otros terrenos, como, por ejemplo, la ecdótica. Durante el desarrollo del trabajo, nos percatamos de que el análisis automático de una obra produce efectos reconocibles si el texto de origen presenta inconsistencias como sílabas adicionales, disposición errónea u omisión de versos o incluso el intercambio de líneas. Si bien esta tarea puede resolverse  mediante una lectura atenta, no cabe duda de que reconocer tales irregularidades con apenas un vistazo supone una ayuda en el examen del texto y su transmisión\footnote{Esta observación se la debo Álvaro Cuéllar y Simon Kroll.}. Confiemos en que la situación haya cambiado en los últimos tiempos y no venga a cuento repetir aquello de que «las ``nuevas tecnologías'' están pasando de largo por la Filología Española sin que la vieja dama, en lo que a la edición crítica se refiere, se inmute», como se lamentaba Ramón Valdés \parencite*[i]{valdes2014} no hace siquiera una década.

En realidad, si no nos dejamos deslumbrar por el brillo que desprende toda novedad antes de que el tiempo la cubra de una venerable pátina, vemos que las aspiraciones de esta tesis beben directamente de la idea primigenia de la filología, por mucho que ofrezca nuevos medios para abordar cuestiones eternas. Friedrich Nietzsche, en una lección inaugural celebrada en la Universidad de Basilea en 1869, defendía que la filología es

\blockquote{tanto un trozo de historia, como un trozo de ciencias naturales, como un trozo de estética: historia, en tanto que quiere aprehender en renovadas imágenes las manifestaciones de las individualidades nacionales determinadas, la ley imperante en el devenir de los fenómenos; ciencias naturales, en tanto que anhela explorar el instinto más profundo del hombre: el instinto del lenguaje; estética, finalmente, porque, de entre un conjunto de Antigüedades, eleva la llamada «clásica» con la aspiración y el propósito de  desenterrar un mundo ideal sepultado y confrontarlo al presente como espejo de los modelos clásicos y eternos. \parencite*[pp. 249-250; traducción propia]{nietzsche1982}} 

Así pues, este trabajo parte de la \textit{estética} de los textos porque pretende encontrar patrones poéticos y dramáticos que los identifiquen. Se basa para ello en las \textit{ciencias naturales}, pues es en la lengua donde buscaremos esos patrones. La síntesis de esto faculta para observar la \textit{historia}, hallar leyes que ofrezcan nuevas perspectivas en la descripción del teatro del Siglo de Oro. En ese sentido, nuestro empeño último es poner herramientas a disposición de otros mejores, no tanto para hacer progresar la ciencia filológica, sino para facilitar su progreso. Dicho de otro modo, aspiramos a aliviar al filólogo de escandir millones de versos y relacionar miles de diálogos\footnote{Además, la recolección de datos requiere una sustancial inversión de tiempo y recursos \parencite[41]{ehrlicher2019}, ambos exiguos en la investigación, lo que hace difícilmente viable la extracción manual de la información dramática y métrica de un corpus de cierta envergadura.} para que pueda dedicar las horas así ganadas a interpretar la evidencia.

Valga la prevención de que los análisis de lectura distante\index{lectura distante} que, entre otros usos, pretendemos facilitar con este trabajo no sustituyen a los métodos tradicionales, sino que ofrecen una perspectiva complementaria. De la misma manera que la placa de Petri del microbiólogo tiene su ámbito de utilización y no suple al microscopio, la lectura distante tampoco reemplaza el escrutinio minucioso de los textos. Ni existe un algoritmo capaz de siquiera aproximarse al detallismo del lector humano ni existe humano con la velocidad de cálculo de un computador. En definitiva, empleémonos a  fondo con las lecturas que lo requieran —\nolinebreak y, a ser posible, también lo merezcan\nolinebreak—\nolinebreak, sin renunciar a una visión general precisa que comprenda el mayor número de elementos del conjunto.

\blockquote{\begin{english}Digital Humanities is an extension of traditional knowledge skills and methods, not a replacement for them. Its distinctive contributions do not obliterate the insights of the past, but add and supplement the humanities' long-standing commitment to scholarly interpretation, informed research, structured argument, and dialogue within communities of practice.\end{english} \parencite[16]{burdick2012}}.

\section{Hipótesis de trabajo, objetivo y preguntas de investigación}
Partimos de una hipótesis sencilla e intuitiva: es posible descomponer la obra teatral considerando sus partes constitutivas y organizarla como datos estructurados\index{datos}, tanto a un nivel superficial dramático como en la profundidad de la forma poética de cada uno de sus versos. De esta manera, se aplica la premisa recursivamente: una vez descompuesta la pieza en sus partes primarias, deberíamos poder hacer lo mismo con cada uno de sus componentes. Finalmente, asumimos que esto puede efectuarse automáticamente dado un conjunto de reglas adecuado.

Por lo tanto, el objetivo es de este trabajo es establecer dichas reglas, definir un procedimiento formal para extraer la información del texto y organizarla de tal modo que pueda ser usada directamente en análisis digitales. Responderemos a las siguientes preguntas de investigación para alcanzar esa meta:
\begin{enumerate}[label=\roman*.]
	\item¿Cómo automatizar el análisis de la estructura dramática?
	\item¿Cómo escandir automáticamente el verso dramático?
	\item¿Cómo organizar la información resultante en un corpus\index{corpus}  anotado?
\end{enumerate}

La empresa plantea ciertas dificultades por el propio género literario del que se ocupa. Por un lado, los estudios de lectura distante\index{lectura distante} tienden a concentrarse en la prosa, por lo que no consideran la particular estructura del texto dramático, lo que se hace más notable si este está en verso. La escansión métrica digital, por su parte, está enfocada en la lírica, por lo que suele abordar estrofas\index{estrofa} predefinidas, mientras que el teatro es polimétrico y no hay una correspondencia entre línea y verso necesariamente ni solución de continuidad delimitando formas poéticas adyacentes.

\section{Estado de la cuestión}
La década de 1980 vio dar los primeros pasos a la filología digital aplicada al teatro español del Siglo de Oro, pero esta no comenzó a adoptar una configuración definida hasta el siguiente decenio. Aún hubo que esperar otros diez años más para que empezara a considerarse realmente popular, ya entrado el siglo \textsc{xxi}. Durante este último milenio, han surgido una plétora de ediciones digitales, bases de datos y programas de análisis literario \parencite{kroll2019b}. La ecdótica digital áurea bebe del proyecto pionero dirigido por Marco Presotto que fructificó en una ambiciosa edición de \textit{La dama boba} \parencite{vega_boba}. Esta trabajo marcó el sendero que habría de seguirse en el futuro, si bien posteriores iniciativas se centran en aspectos concretos: aquella edición estaba concebida como un laboratorio de pruebas, por lo que, a pesar de ser deseable, resultaría muy costoso explorar un corpus grande de la misma manera\footnote{José Manuel Fradejas Rueda \parencite*{fradejas2022a} pone en perspectiva histórica los  métodos y formatos empleados hoy en la edición crítica digitalizada.}.

El propio Presotto \parencite*[32-36]{presotto2018} nombra algunas iniciativas loables. El grupo Prolope, por ejemplo, pone a disposición del público tanto el texto crítico como la edición de una selección obras del Fénix \parencite{prolope2023}, pero no solamente eso, sino que su forma de abordar la edición constituye «un instrumento  imprescindible  que  ha  significado  un cambio de paradigma en los estudios dedicados al Siglo de Oro» \parencite[2]{presotto2023}. El grupo \ac{griso} ofrece los textos de las ediciones de los autos de Calderón que ha publicado en Reichenberger.  Moretianos \parencite{moretianos2023}, responsabilidad del grupo PROTEO, se dedica a la obra de Agustín Moreto. Asimismo, otros proyectos, como Cervantes Virtual \parencite{cvc2021} o Canon 60 \parencite{tc12} garantizan el acceso a textos de calidad de diversas procedencias, como los producido por el GIC \parencite{gic2023}. Además, él mismo codirige Gondomar Digital junto a Luigi Giulani y Fausta Antonucci \parencite{gondomar}.

Las bases de datos también han gozado de gran predicamento, tanto en sus variantes dedicadas a compilar y organizar obras en crudo, información sobre estas u otras cuyo objeto es la información metatextual, pero sin anotar directamente el texto. Bajo el paraguas ASODAT \parencite{ferrer2023}, se encuentran, entre otras, algunas tan asentadas como Artelope \parencite{oleza2013}, Calderón Digital \parencite{antonucci2018}, CLEMIT \parencite{urzaiz2020}, DICAT \parencite{dicat}, Digital Música Poética \parencite{dgp}, ETSO \parencite{cuellar2018}, CATCOM \parencite{ferrer2013} o MANOS \parencite{greer2022}. Gracias a la combinación de estas últimas, por ejemplo, ha podido seguirse la pista del ejercicio de la composición dramatúrgica de autores no tan tratados como los grandes \parencite{garciareidy2019}\footnote{Además, habría que señalar bases de datos que, sin referirse directamente a la literatura española o siquiera a la temprana Edad Moderna, albergan información relevante para el estudio de las letras áureas. Valga para ilustrarlo la que recoge la actual ubicación de los fondos —hoy dispersos— de la biblioteca privada del romántico alemán Ludwig Tieck, quien, aparte de dramaturgo, fue un ávido coleccionista de sueltas de dramas áureos españoles \parencite{tieck2018}.}. De aparición más reciente, AuTeSO \parencite{auteso}.

Resultan mucho menos numerosos los trabajos que abordan el análisis del propio texto. No obstante, encontramos algunos como los que se ocupan de la estilometría \parencites{cuellar2018}{hernandezetal2022}. Precisamente, el trabajo de
\citeauthor{cuellar2023}~\parencite*{cuellar2023} se valió de esto, entre otras cosas, para la celebrada atribución lopesca de \textit{La francesa Laura} \parencite{lope_laura}. Estos se valen del módulo de estilometría\index{estilometría} Stylo \parencite{eder2016} del lenguaje de programación R\index{R} para realizar pruebas de autoría. Por la propia concepción de los análisis estilométricos tradicionales\footnote{Existen más aproximaciones que han demostrado una precisión semejante a los análisis estilométricos tradicionales basados en la frecuencia de uso de palabras. Ver \citeauthor{kroll2022a}~(\cite*{kroll2022a} y \cite*{kroll2023}) y \citeauthor{plechac2018}~\parencite*{plechac2018}.}, el texto teatral áureo se coteja considerando solo sus parlamentos y, estos, como si de prosa se tratase. Esto, por otra parte, ha obligado a desarrollar métodos de tratamiento de las fuentes, lo que llevó a Cuéllar a componer sus propios modelos lingüísticos  automatizar transcripciones paleográficas \parencite{cuellar2023b}. Otro tipo de información textual más allá del léxico no ha despertado tanto interés en el ámbito hispánico, si bien, recientes estudios orientados a la poética abren una puerta para futuras aproximaciones \parencite{hernandez2022a}.

Curiosamente, la descripción cuantitativa de la obra teatral cuenta con bases teóricas que se remontan atrás más de medio siglo. Solomon Marcus~\parencite*{marcus1973}, por ejemplo, ya elaboró en su día un método preciso para representar la presencia escénica de los personajes y sus relaciones. Esta metodología se engloba en lo que Romanska \parencite*[446]{romanska2015} denominó más tarde \textit{dramétrica}\index{dramétrica}, que se traslada directamente al modelo digital. Contando con los textos debidamente anotados, pueden resolverse hoy en pocos segundos trabajos que le habría llevado semanas a Marcus. Para el análisis del teatro áureo, sin embargo, esa aproximación presenta un inconveniente: toma como segmento básico la escena. El drama del Siglo de Oro carece de tal división, al menos de forma explícita. En todo caso, en este trabajo no dilucidaremos la cuestión, pero sí propondremos algunos medios con nuestros resultados para aligerar, hasta cierto punto, la labor de aquellos dispuestos a lidiar con ella.

Existen otros análisis dramáticos que no están sujeto a la misma indeterminación. Un ejemplo de ello lo ofrece Ilsemann \parencite*{ilsemann1998}, a pesar de que no aborda el teatro español áureo, sino el inglés isabelino —\nolinebreak la frecuencia y extensión de los parlamentos de varias piezas de Shakespeare\nolinebreak—\nolinebreak. Las variables que describe son comunes a todas las obras teatrales, cualesquiera que sean su lengua y época, por lo que estos estudios del drama constituyen un magnífico modelo para aplicar a otras lenguas, sea en sus métodos \parencite{ilsemann2013} o sus herramientas \parencite{wilhelm2013}.

 En otras lenguas encontramos aproximaciones similares, si bien más tímidas. El teatro alemán, por ejemplo, ha sido investigado de forma cuantitativa \parencite{dennerlein2015} y, hoy en día, cuenta con herramientas libremente accesibles para obtener información dramétrica de los textos \parencite{schmidt2019}; solo muy recientemente se han hecho incursiones en el teatro español, precisamente en el ámbito del drama aurisecular \parencites{ehrlicher2020}{lehmann2022}.

Todos estos estudios, como hemos dicho, requieren fuentes anotadas con la información necesaria y estructurados en un formato adecuado. Unos sesenta textos se encuentran disponibles en la colección Canon 60 \parencite{tc12} como páginas web, aparentemente creados a partir de una fuente en \ac{xmltei}, que, no obstante, necesitan cierto tratamiento previo para estandarizarlos. Más ambicioso es el proyecto DraCor \parencite{fischer2019}, que compila diversos corpus teatrales en once idiomas. Aparte de los generales de las lenguas inglesa y española, dispone de secciones específicas para Shakespeare y Calderón, a la última de las cuales ha contribuido Sound and Meaning con obras producidas mediante los resultados expuestos en esta tesis. Al estar codificado en el estándar de anotación \ac{xml} de textos literarios de la \acl{tei} (\ac{tei}), facilita diferentes análisis dramáticos. El trabajo de \citeauthor{ehrlicher2020}~\parencite*{ehrlicher2020} —además, responsables en su mayor parte de la sección calderoniana CalDraCor—, por ejemplo, emplea este tipo de archivos como corpus. Todos ellos tienen en común la obligada necesidad de incrementar sus fondos al lento paso que obliga el procesamiento de los textos.

La escansión automática de poesía española obtuvo sus primeros resultados relevantes gracias al estudio desarrollado por \citeauthor{gervas2000}~\parencite*{gervas2000}. Este toma las palabras del verso e intenta separar sus sílabas, bien utilizando un \textit{diccionario}\footnote{Se denomina \textit{diccionario} a un tipo de datos consistente en un conjunto de elementos, cada uno de los cuales consta de una \textit{clave} y un \textit{valor}. Se accede al valor mediante una referencia a la clave. Piénsese en un diccionario de consulta al uso: la definición de una voz es el valor del elemento y el lema al que alude, su clave. En el caso mencionado, la clave es una palabra y los datos la lista de sus sílabas.}, bien mediante reglas cuando este no contiene una clave para la palabra en cuestión. De esta forma, obtiene el número de sílabas, una lista de estas, los acentos internos de las palabras y la rima del verso. No obstante, este método presenta algunos aspectos susceptibles de mejora. El primer problema es consecuencia de la identificación de acentos considerando solo la ortografía. Esto hace que determinantes átonos como \textit{la} o \textit{mi} sean considerados como tónicos. Por lo tanto, el primer verso del celebérrimo soneto de Quevedo «Amor constante más allá de la muerte», que es el ejemplo que ofrece Gervás, se interpreta como \textit{ce rrár po drá mís ójos lá pos tré ra}. A efectos de la clasificación de Gervás, este verso no sería problemático, pues solamente considera endecasílabos\index{verso!endecasílabo} puros de las cuatro clases principales, por lo que, en la práctica, ese heroico corto se clasificaría igual que uno largo. Sin embargo, cabría preguntarse cómo resolvería versos con acentos impropios\index{acento!impropio}, como los que aparecen en ese mismo soneto. El método de resolución de sinalefas\index{sinalefa} también presenta inconvenientes, pues estas se abordan secuencialmente hasta que el verso produce el recuento silábico deseado. Si el número de sinalefas coincide exactamente con el de sílabas supernumerarias, la resolución no plantea dudas, pero si hay más sinalefas potenciales que efectivas, es posible que atacar el verso de izquierda a derecha emplace la sinalefa entre las sílabas equivocadas. Asimismo, Gervás reconoce en el artículo que su programa no provee un método satisfactorio para resolver ambigüedades derivadas de figuras como sinéresis o hiatos. Esta aproximación rinde una precisión del $88{,}73$~\% de acuerdo con el propio Gervás.

Tales carencias no se le pasaron por alto a \citeauthor{navarrocolorado2015}~\parencite*{navarrocolorado2015,navarrocolorado2017}, quien desarrolló una alternativa mucho más precisa orientada a los versos endecasílabos. En primer lugar, etiqueta gramaticalmente las palabras para decidir si llevan acento métrico o no. Seguidamente, procede a la división silábica y detección de la sílaba tónica. Se decide qué palabras acarrean acento prosódico en función de la etiqueta gramatical. Si el recuento de sílabas es igual a once, se da por resuelto el verso; si no, se aplica el módulo de ajuste y, si hay varias alternativas, el de desambiguación. Esta última se hace atendiendo a la frecuencia de las posibles alternativas en un corpus. De esta manera, solventa el problema mencionado con la distinción de palabras átonas y tónicas del que adolecía el analizador de Gervás. Por otra parte, las ambigüedades ya no se resuelven linealmente, sino mediante un método que ofrece una mejora significativa y alcanza un 94{,}44~\% de precisión en endecasílabos \parencite[51735]{marco2021}.

A las iniciativas pioneras, hay que añadir la de \citeauthor{agirrezabal2017}~\parencite*{agirrezabal2017}, que usa redes neuronales para mejorar la aproximación de Gervás hasta el {90{,}84} \% de precisión. En los últimos años, se han añadido otras dos, Rantanplan \parencite{rosa2020} desde una perspectiva clásica con un 96{,}22~\% de aciertos \parencite[51740]{marco2021} y Jumper \parencite{marco2021b}, que aplica un novedoso método que prescinde de la silabación, con una precisión del 94{,}97~\% \parencite[51740]{marco2021}. Durante las últimas correcciones de esta tesis se presentó ALBERTI \parencite{rosa2023}, un sistema de clasificación basado en \acs{bert}, cuyo módulo de predicción de patrones versales alcanza {91{,}15} ~\% de precisión mediante redes neuronales.

A pesar de los excelentes resultados en verso fijo, la mezcla de versos presenta más complicaciones, pues el número de versos correctamente escandidos cae en Jumper hasta el 81{,}73~\%, en Rantanplan hasta el 78{,}67~\% y el 49{,}38~\% con el analizador de Navarrro-Colorado \parencite[51740]{marco2021}.

\section{Metodología y estructura del trabajo}
Este trabajo comienza enfocando el objeto del estudio desde una perspectiva propiamente deductiva. Parte de las generalizaciones de diversas ramas de la teoría literaria para codificarlas formalmente y aplicar esta formalización a un corpus de prueba. A la luz de los resultados de la primera aproximación, afinaremos inductivamente las reglas en situaciones excepcionales. En otras palabras, si se dan casos concretos que se desvían de lo previsto, existe la posibilidad de modificar el modelo considerando esas instancias para que el analizador automático sea capaz de identificarlos y tratarlos en consecuencia durante futuros análisis.

Hemos dividido el trabajo en dos partes, cada una con cuatro capítulos distribuidos de forma paralela. La primera consiste en una presentación descriptiva de los aspectos teóricos. Provee al lector de un compendio de los conceptos requeridos para el desarrollo de la segunda parte. Esta, a su vez, consiste en la formalización algorítmica de los aspectos teóricos vistos y la descripción de los tipos de datos abstractos empleados para estructurar la información textual y metatextual. Ambas partes están organizadas según la misma disposición: en primer lugar, se define el marco y, dentro de este, se va refinando la aproximación desde los trazos más gruesos a las pinceladas más minuciosas. Así lo haremos tanto para presentar el fondo teórico y metodológico como para formalizarlo. Partimos de una perspectiva general y vamos aumentando el nivel de detalle: drama, métrica y fonología; y, en estas, a su vez, de lo general a lo particular.

Empezamos por ubicar el estudio dentro de las humanidades digitales. Este trabajo está orientado a filólogos, a los que suponemos familiarizados con la métrica, la fonología y el drama; sin embargo, es probable que no lo estén tanto con las humanidades digitales\index{humanidades digitales}\footnote{Al menos, si contemplamos los planes de estudio filológicos de las universidades austriacas y españolas al momento de redactar estas líneas.}. En previsión de diferentes grados de experiencias con lo digital, el capítulo dedicado a esta relativamente nueva disciplina no solo aborda aspectos a tener en cuenta en la segunda parte del trabajo, sino que presenta una breve introducción a la historia y naturaleza de esta rama humanística. Una vez hechas las presentaciones, definimos los principios ontológicos que emplearemos. Esto es, determinaremos las categorías a buscar y, para ello, debemos conocer los entes susceptibles de ser objeto de estudio así como las interrelaciones que establecen entre sí. Tomaremos prestadas para ello algunos conceptos de la \textit{teoría de la información} para concretar la idea de \textit{datos}\index{datos} y sus diferentes tipos porque es la manera en que manejaremos las distintas categorías.

Una vez establecidas estas premisas, disponemos del marco de referencia en el que ubicar las partes concretas de la obra teatral y así identificar y organizar sus componentes. Para ello, en el siguiente capítulo de esta parte, nos referiremos al drama y la dramétrica\index{dramétrica} para establecer los elementos estructurales susceptibles de ser materia de estudio cuantitativo; así, determinamos las entidades a caracterizar a la hora de elaborar el modelo formal y, más importante, cómo organizarlas. En la consecución de este objetivo, además, hemos de caracterizar no solo qué entidades deseamos aislar como abstracciones, sino también cómo se plasman estas en el texto impreso en la práctica  o, en nuestro caso, en la edición digital. Aparte de las particularidades estructurales, emplearemos este capítulo para acercarnos a los aspectos textuales del corpus que tomamos como modelo. Consideraremos, pues, las convenciones editoriales de las ediciones modernas usadas y discutiremos la elección de estas.

Cuando seamos capaces de descomponer el texto dramático en sus partes, deberíamos centrarnos de manera particular en una de ellas: el verso. Sin embargo, el teatro no se compuso para ser leído, sino representado ante el público\index{público}. Por lo tanto, analizar sus versos teniendo en cuenta tan solo su grafía, muy especialmente cuando, como en el caso que nos ocupa, se pretende proporcionar datos sobre su ritmo y sonido, resulta una solución no del todo satisfactoria. Por este motivo, hemos optado por abordar los versos desde una perspectiva fonológica, lo que tratamos en el tercer capítulo de la primera parte. Emplearemos la fonética articulatoria como marco teórico para representar los versos en la formalización, por lo que nos centraremos particularmente en ella, pero introduciremos algunas nociones de fonética acústica cuando la ocasión lo requiera, que resultarán de gran utilidad para que dirimir por la vía experimental ciertas ambigüedades. 

En el último capítulo de la parte teórica, nos ceñiremos a las prescripciones establecidas por la métrica española. Por este motivo, tomaremos de las poéticas aquellas descripciones que permitan definir reglas en la segunda parte tanto para interpretar los versos como para agruparlos en unidades superiores, como estrofas\index{estrofa} o poemas\index{poema}. 

La segunda parte de este trabajo se ocupa de formalizar los elementos teóricos presentados en la primera. No obstante, antes de entrar de lleno, comenzaremos por explicar algunos aspectos metodológicos y técnicos de la implementación del modelo de pruebas y las herramientas empleadas, así como por justificar la elección de algunas de estas, tales como el lenguaje de programación Python\index{Python} o unas aplicaciones informáticas auxiliares, en lugar de otras de funcionalidad equivalente. La decisión de no ubicar este capítulo al final es deliberada: en los subsiguientes no entraremos en el diseño material que usamos en el computador, sino que presentaremos algoritmos y remitiremos al apéndice para ver su materialización en código informático. Sin embargo, algunos de los aspectos del  diseño \textit{en papel} se entienden mejor conociendo las cortapisas impuestas por la programación para tratar textos digitales en un entorno real, con sus limitaciones y particularidades. Esto es, aunque el código de los apéndices es una plasmación de los algoritmos propuestos, estos se concibieron con la idea de ser trasladados a un entorno computacional. Asimismo bosquejaremos una visión general del mecanismo de análisis textual, de manera que se aprecie cómo se unifica lo expuesto por separado en los siguientes capítulos, cómo interaccionan entre sí los diferentes módulos descritos.

Comenzamos por analizar la manera de transformar en datos el texto dramático, tal y como lo encontramos en su edición digital corriente, para organizar la información constituyente de modo que acepte el tratamiento digital de manera sencilla, tan directa como sea posible. Con el texto así preparado, comenzaremos el análisis automático para identificar, extraer y organizar como datos \textit{estructurados}\index{datos!estructurados} la información susceptible de estudio cuantitativo. Este paso, si bien instrumental para nuestro propósito, ofrecería ya un resultado finalista si el objetivo es codificar la información textual de comedias sin enriquecerla con otra derivada de análisis adicionales, como pudiera ser el metro de cada verso. Los resultados se ven en las adiciones a CalDracor del proyecto Sound and Meaning, codificadas a \ac{xmltei} mediante la implementación práctica de la aproximación que se presenta en los capítulos sexto y séptimo.  

La escansión automática emplea textos transcritos fonológicamente, por lo que, antes de ponernos con ella, nos ocuparemos en el siguiente capítulo del módulo aparte que aborda la tarea. Para afrontar la transcripción fonológica, trataremos, por una parte, la transformación mediante reglas de los grafemas en fonemas y, por otra, las de división silábica.  Asimismo, presentaremos los resultados de las pruebas realizadas para determinar la precisión del sistema.

Finalmente, acometeremos la escansión métrica automática. Comenzaremos dando una visión general del procedimiento. Seguidamente, iremos deteniéndonos en los módulos que la desarrollan para examinar su funcionamiento en detalle. Esto es, de qué manera se aplica la teoría métrica mediante reglas lógicas y qué casos aconsejan introducir excepciones como respuesta a los resultados. En la evaluación y pruebas, emplearemos obras de la colección de autos de Calderón de la Barca editados por el \ac{griso} \parencite*{griso2020} y el corpus de sonetos\ac{adso} \parencite{navarrocolorado2016b}. Este último se ha empleado en los estudios más recientes como \textit{gold standard} para la evaluación de la escansión automática de sonetos \parencite[51735]{marco2021}.

Al completar ambas partes, estaremos por fin en condiciones de volver a las preguntas de investigación arriba formuladas, discutirlas según las vicisitudes encontradas a lo largo del periplo que aquí comienza y plantear respuestas de la mejor manera posible.

Finalmente, hemos incluido cuatro apéndices. El primero contiene un resumen de la tesis en alemán para satisfacer los requerimientos de la Universidad, así como su traducción española e inglesa. El segundo apéndice ofrece una implementación de lo presentado en esta tesis al lenguaje de programación \textit{real}, con el código Python empleado para someter a prueba la validez de nuestra aproximación. El tercero describe de forma sucinta el uso de los programas desarrollados a partir de las ideas aquí expuestas, tanto para sugerir aplicaciones posibles como para demostrar la sencillez de su uso. El último apéndice consiste en un análisis cuantitativo de la primera jornada de \textit{La vida es sueño}, que ilustra mediante el ejemplo las demandas de potenciales estudios de esa índole.