\chapter{Conclusiones}
\epigraphhead[50]{\epigraph{Y pues que yo soy criado\\
		de paz, solamente os ruego\\
		que consideréis, señores,\\
		que de los yerros ajenos\\
		no hay cosa como callar;\\
		perdonadnos, pues, los nuestros.}{Calderón, \textit{No hay cosa como callar}}}

\section{Recapitulación}
Las humanidades digitales\index{humanidades digitales} han demostrado su valía a lo largo de las últimas décadas y, desde el mismo nacimiento de la disciplina con el \textit{Index Thomisticus}\index{humanities computing@{\textit{humanities computing}}}, ha sido la filología el ámbito de estudio en el que más presentes se han hecho. Esto ha propiciado unas cotas de desarrollo incomparables con otras ramas humanísticas, en las que la aparición de métodos asistidos por computador ha sido más tardía. En cualquier caso, la computación aplicada a la filología no es solamente viable hipotéticamente, sino una realidad consolidada para abordar el examen de  grandes corpus literarios y, de ahí, la idea que puso en marcha esta empresa.

Disponemos de herramientas que permiten escudriñar detalles en un inmenso volumen de texto. Podemos realizar en cuestión de minutos tareas que hubieran sido inconcebibles hace ocho décadas y económicamente inalcanzables treinta años atrás. Pretender que la ciencia filológica permanezca ajena al desarrollo tecnológico y científico —ajeno a la sociedad de su tiempo, al fin y al cabo— resultaría no solo tremendamente arrogante, sino que haría un flaco servicio a la disciplina, al privarla de usar todos los medios disponibles para la interpretación textual, sin importar si son digitales o tradicionales. El científico, también en su vertiente humanista, no se debe al método, sino al conocimiento.  

Esto no quiere decir que la filología deba renunciar a su tradición hermenéutica, ni mucho menos. Muchas de las diversas escuelas de crítica literaria basadas en \textit{close reading} tienen en su haber sólidos argumentos que las avalan como marco interpretativo. Esto, sin embargo, no debería estar reñido con abrazar los avances técnicos, tanto para complementar y dar más peso a las aproximaciones clásicas como para destapar nuevas cuestiones que han permanecido ocultas hasta ahora, enterradas bajo una colosal montaña de fuentes, una mole tan descomunal que hacía humanamente imposible acercarse a los aspectos mínimos desde la totalidad textual. No se tome esto como un llamamiento a aceptar nuevas aproximaciones a despecho de la rigurosidad, pues esta debe ser clave para cualquier metodología, sea asistida por computador o no. Siempre sin olvidar que, en el centro de todo ello, se ubica el texto.

La práctica consecuente de la filología digital requiere, no obstante, conocer sus cualidades y limitaciones. Podemos usar una metodología \textit{asistida} por datos para evaluar cierta información textual y emitir un juicio, considerar hipótesis o plantear nuevas cuestiones. Podemos, por otro lado, decidirnos por una metodología \textit{guiada} por datos para obtener respuestas cuantitativas a nuestras cuestiones. En nuestro caso, hemos definido de antemano tanto las preguntas como los criterios exactos bajo los que considerar la solución, así como las condiciones objetivas en las que la buscaremos. De cualquier manera, esto ha condicionado tanto el tipo de datos empleados como los nuevos tipos propuestos.

En efecto, el modelo elegido ha determinado en gran medida el tipo de datos requeridos: no estructurados, semiestructurados  o estructurados. En los dos últimos casos, se han tenido que reunir todos aquellos aspectos del texto necesarios para el estudio al que debía servir, las investigaciones del proyecto Sound and Meaning. En el caso de los datos no estructurados, hemos descrito ciertas herramientas más o menos complejas para poder utilizarlos de forma masiva en, por ejemplo, búsquedas de fragmentos textuales. Esto ha sido quizás una de las precondiciones que más peso ha tenido nuestro trabajo. Para satisfacer las demandas de ambas aproximaciones, ha sido necesario concebir una estructura de datos que permitiera estudios cuantitativos completamente digitalizados, pero que también proporcionara información útil para hacer inferencias cualitativas. En otras palabras, por una parte, los datos producidos deben ser aptos para su proceso inmediato mediante, por ejemplo, programas de estadística o \textit{machine learning}\index{machine learning@\textit{machine learning}} y, por otra, han de permitir las búsquedas casuísticas particulares según unos criterios dados.

Para ello, hemos diseñado un sistema de datos estructurados. Este representa elementos textuales del drama y de la métrica, así como otros editoriales que permiten identificar la procedencia del texto adecuadamente. Hemos considerado también la eventualidad de que los textos se presenten en diversas codificaciones, por lo que, antes de tratar una fuente, nos hemos asegurado de unificar la forma de representación para que todos ellos estén dispuestos tal y como el programa espera encontrarlos.

A la hora de formalizar los presupuestos teóricos y, sobre todo, pensando en su modelización ulterior con un lenguaje de programación real, hemos tenido en cuenta que, dado el tamaño del programa, habíamos de atenernos no solo a las reglas sintácticas propias de un lenguaje concreto, sino a unas buenas prácticas generales para mantener una complejidad asumible y evitar ralentizar el desarrollo. Dicho de otra forma: además de que funcione, hay que poner todas las facilidades para llegar a ello de una manera eficiente, rápida y segura.

En lo concerniente al teatro, hemos presentado los elementos necesarios para describir la estructura de la obra dramática. Lo hemos acometido partiendo del hecho teatral, de su posición  en el mapa de los grandes géneros literarios, ofreciendo casi una panorámica de su esencia por contraste con la narrativa y la lírica. Una vez ubicado nuestro objeto de interés, hemos repasado sus características desde la dramatología para sentar las bases que han permitido descomponer la obra en sus elementos estructurales.

De esta visión general, hemos reducido el foco a la vertiente de la obra fijada por escrito, ya que esta es el objeto final de nuestra investigación. Aquí hemos revisado la relación de la estructura externa del texto con aquellos elementos escénicos y ortotipográficos mensurables que sirven de criterio para descomponerlo. Hemos discutido la forma de reconocer dichos elementos y hasta qué punto son suficientes sus criterios de identificación inferidos de la estructura externa  para determinar aquellos sin necesidad de recurrir a la interna. Precisamente, definir los constituyentes que permiten asignar valores a estas variables es la parte fundamental de nuestro análisis digital del drama.

Dado que nos centramos en el texto escrito, presentamos las convenciones que este sigue, prestando especial atención a las ediciones críticas modernas y sus particularidades, pues son estas las que más garantías ofrecen en cuanto al respeto por las fuentes originales\footnote{En sentido estricto, no nos ocupamos de la edición crítica sino del texto fijado por esta.}.

De todo esto, hemos identificado los segmentos más productivos, partiendo de elementos estructurales dramáticos mínimos, como son los parlamentos y las acotaciones externas. Si bien hemos podido marcar los actos, que están indicados de forma explícita, no ha sido así con secuencias de orden inmediatamente inferior. No obstante, nuestros resultados ofrecen información que facilita su identificación. En concreto, mediante el seguimiento de la entrada y salida de personajes a escena para tratar de aproximar los puntos en los que el escenario queda vacío. Asimismo, la segmentación métrica que hemos conseguido ayuda al análisis, si bien no todos los cambios métricos denotan una secuencia englobante y se requiere evaluar los elementos internos.

Se han identificado los elementos que caracterizan al parlamento. Por un lado, sus atributos, que son el locutor o locutores que lo pronuncian, al igual que elementos modales opcionales, por ejemplo, indicaciones escénicas de aparte o cantado. Por otro, sus partes constituyentes o líneas, que también están indicadas explícitamente. Mientras que en las ediciones históricas la línea se corresponde con el verso y varios parlamentos pueden compartir una línea, este no es el caso en las modernas. Por fortuna, no ha supuesto gran problema porque los versos compartidos se señalan por medio del sangrado en la mayoría de las ediciones empleadas.

Hemos presentado los fundamentos para la escansión métrica de los versos teatrales y su  ordenación en agrupaciones estróficas. Estos no difieren de los empleados en la lírica, por lo que, a todos los efectos, nos hemos conducido por las mismas reglas que se aplican a los poemas\index{poema}. Por ello, los tratados de poesía han sido el punto de partida para examinar las configuraciones métricas de las obras dramáticas. Nos hemos centrado en los metros más comunes, como son los octosílabos, endecasílabos, heptasílabos y hexasílabos, pero sin descuidar la posibilidad de versos quebrados ni la presencia de otros menos habituales, como los pentasílabos, siempre teniendo en cuenta que otras medidas son igualmente plausibles. El límite lo marca la prosodia española, por lo que los pasajes cantados pueden no corresponderse con el análisis, al atenerse a un ritmo acentual.

La unidad autosuficiente mínima de la métrica castellana es el verso, de ahí que nos hayamos ocupado de caracterizar la configuración que adoptan componentes subordinados. En concreto, las partes elementales tratadas son la sílaba y el axis métrico. Hemos estimado el tamaño y ritmo del verso, considerando los elementos definitorios de la sílaba métrica, como los fenómenos intersilábicos y la tonicidad. También hemos identificado la naturaleza y posición de las vocales de sílabas en contacto capaces de alterar la cantidad del verso y el axis métrico donde se resuelve la rima. Este es un criterio fundamental para combinar un verso con otros, puesto que altera el cómputo silábico respecto al gramatical según las vocales que queden a la derecha del eje. Abrimos también la posibilidad de clasificar versos de una cantidad dada, tanto endecasílabos en sus tipos tradicionales como octosílabos, conforme a diversas tipologías.

Los versos se asocian entre sí de acuerdo con su metro individual y rima  en agrupaciones tanto estróficas como no. En el primer caso, hemos discriminado series de versos para formar una unidad intermedia, la estrofa\index{estrofa}. Esta es simple, como el cuarteto, o compuesta si está constituida por varias estrofas, como la décima, por ejemplo. Las estrofas se agrupan de forma que el conjunto constituye una unidad de orden superior, como un poema\index{poema}. Sin embargo, la configuración intermedia no es una condición necesaria para la creación de una unidad poética de orden superior porque los propios versos también se agrupan independientemente para constituirla, como sucede en el caso de los romances. Esto lo hemos tenido también en cuenta.

El análisis teatral obliga a buscar patrones para, en primer lugar, aislar las diferentes unidades estróficas y superiores que componen la obra, dado que, al contrario de lo que ocurriría con un poema\index{poema}, las agrupaciones métricas se suceden en esta sin solución de continuidad. Además de la rima, hemos observado el metro de los versos, dado que las estrofas heterométricas\index{estrofa!heterométrica} no son arbitrarias, sino que suelen combinar tipos particulares de verso, como heptasílabos y endecasílabos. Sin embargo, el conjunto de combinaciones suele limitarse a unas pocas en la práctica. Disponer de estos segmentos métricos superiores como elementos individuales ha permitido clasificar las agrupaciones según los modelos vistos en la teoría.

Finalmente, hemos abordado la métrica con una perspectiva fonológica. Esto ha permitido representar textos escritos considerando los aspectos sonoros intrínsecos que de ellos se desprenden y aproximarlos a la pronunciación. Sin dejar de ser una idealización del habla humana, la fonología ha facilitado hacer una distinción clara entre las unidades textuales. Esto es, empleándola, podemos definir un modelo del signo lingüístico en la lengua oral. Esta idealización nos abstrae de las características físicas de las realizaciones particulares de los sonidos. Así, resulta factible obtener una generalización lingüísticamente relevante de la pronunciación del texto escrito. En tanto que partimos de  textos escritos y no de muestras de habla producidas por un hablante, hemos transcrito —más bien transliterado— las fuentes. 

Para construir este modelo, hemos vuelto sobre el concepto de fonema, que hemos tomado como unidad mínima de los textos. Lo hemos hecho, claro está, considerando solo el repertorio fonológico del español. La aproximación ha sido principalmente articulatoria porque nos interesaba representar con precisión y de forma concisa las unidades fonológicas que se infieren del texto. Sin embargo, ha sido obligado considerar algunas cuestiones fonéticas para resolver sistemáticamente determinadas secuencias de fonemas.

Por conveniencia, hemos entrado someramente en algunos aspectos de la fonética acústica. Si bien no la hemos usado en la representación del texto, ha sido una potente herramienta para evaluar muestras de lengua hablada como vía para resolver indeterminaciones. En caso de ambigüedad, hemos evitado recurrir a la autoelicitación; en su lugar, hemos observado textos de informantes para inferir la realización más neutra o, al menos, si es aceptable una forma determinada. Esto es, cuando el camino deductivo parece llegar a su fin, siempre podemos tomar el desvío inductivo mediante la fonética acústica. Lamentablemente, esto también significa que, ante la presencia de varias alternativas válidas, hemos descartado todas excepto una para poder generalizar. Como ya dijimos antes, en la creación del modelo siempre se pierde algo respecto al referente real.


\section{Respuestas a las preguntas de investigación}
Habíamos comenzado este trabajo formulando tres preguntas de investigación: cómo automatizar el análisis de la estructura dramática, cómo escandir el verso dramático y cómo organizar toda esta información en un corpus. Veamos una por una y pormenorizadamente las respuestas que proponemos.

\subsection{I. ¿Cómo automatizar el análisis de la estructura dramática?}
En lo referente a la primera pregunta de investigación, hemos visto en los capítulos sexto y séptimo que la estructura dramática puede organizarse en función de sus elementos constituyentes. En concreto, hemos partido de entidades ya propuestas, a la manera descrita por el estándar \ac{xmltei}. Sin embargo, esta disposición no se presta directamente a los exámenes de lectura distante\index{lectura distante} porque, para considerar la complejidad estructural de la pieza dramática, requeriríamos trasladar la información a datos estructurados, esto es, en forma de tabla, que es el formato con el que funcionan internamente los programas estadísticos. Si bien existen paquetes que \textit{leen} datos semiestructurados, como los representados en los susodichos archivos \ac{xmltei}, estos se convierten previamente a un \textit{marco de datos}, o sea, se tabulan. El problema que esto presenta es que las entidades están predeterminadas para tareas muy concretas, bien con información estadística del léxico de cada obra, como hacen los programas de estilometría\index{estilometría}, bien con relaciones escénicas, como es el caso de los paquetes dedicados a los estudios dramétricos. Esto supone una limitación  que dificulta sobremanera los análisis métricos distantes, en tanto que no tiene en cuenta el metro como entidad dentro de la estructura dramática. 

La solución propuesta consiste en codificar la información dramática considerando la línea como la unidad mínima, de modo que los datos adicionales relativos a esta puedan definirse en tantas columnas como se requiera. Para nuestro modelo, hemos planteado entidades de cuatro tipos. Primero, aquellas destinadas a los metadatos, que se aplican a la obra en su totalidad; esto es, en cada una de esas columnas, todas las filas correspondientes a una sola composición teatral contienen un valor idéntico. Este tipo de columnas incluyen información invariable como el autor de la pieza, el título y el subtítulo de esta, el género y el subgénero, el año de publicación, así como la información relevante sobre la procedencia editorial de la fuente. En segundo lugar, tenemos elementos estructurales. Estos son externos, como el número de acto\index{acto} y de línea\index{línea}, o internos, como el número de parlamento. Tercero, tenemos elementos dramáticos, que incluyen al locutor y el propio texto de la línea. Finalmente, hay elementos poéticos; estos son el número de verso y de unidad poética en el que este se enmarca. La disposición tabular con arreglo a estas variables permitiría establecer redes de personajes como en los análisis dramétricos ordinarios, pero también segmentar la obra de forma externa e interna atendiendo a otros.

De esta manera, a cada línea le corresponde un número y, además, el propio texto, independientemente de si este se repite de nuevo en la obra, pues, a efectos de clasificación, serían considerados instancias distintas. Tendría, sin embargo, los mismos metadatos que el resto de la obra, así como el número de jornada del resto de líneas que se pronuncian en el mismo acto y el mismo número de parlamento y locutor que el resto de versos de dicho parlamento. El número de verso puede ser único, pero no necesariamente, ya que a veces se distribuye a lo largo de varias líneas —en teoría, sería posible distribuirlo en tantas como sílabas en las que podemos separarlo—, cuando el verso lo inicia un personaje y otro u otros terminan de recitarlo.

\subsection{II. ¿Cómo escandir automáticamente el verso dramático?}

Para contestar a la segunda pregunta, hemos construido la respuesta sobre la precedente, pues la identificación del verso dramático implica caracterizar primero las partes constituyentes externas del texto teatral. Una vez hecho esto, lo hemos procesado verso a verso considerando el contexto métrico y dramático. Para ello, hemos propuesto un modelo de escansión fonológico recursivo basado en prioridades. La base es el sistema \textit{clásico} hasta ahora aplicado a la lírica, que consiste en un análisis morfosintáctico del verso para distinguir entre categorías gramaticales átonas y tónicas, división silábica y ajuste de las sílabas en función de un conjunto de reglas dado. Sin embargo, cada una de las partes de este enfoque es susceptible de mejoras en el procedimiento para llevar a cabo su propósito. Esto es, dar los mismos pasos empleando diferentes medios. En nuestro caso, hemos perfeccionado la división silábica y el tratamiento de las categorías gramaticales y hemos formulado reglas diferentes para el ajuste silábico. Asimismo, hemos cambiado la forma de representación del texto para trabajar sobre fonemas.

El sistema que planteamos parte de una transliteración fonológica\index{fonología} de las palabras, incluyendo la distinción entre vocales\index{vocal} silábicas y no silábicas\index{vocoide}. La aproximación grafémica\index{grafema} tradicional implica un análisis fonológico inconsciente, por lo que resulta difícil automatizarlo sin recurrir a un proceso fonológico implícito, con la complicación adicional de tener que ofuscarlo bajo subterfugios ortográficos. Trabajar con fonemas, por un lado, simplifica el desarrollo conceptual del método porque permite operar en términos fonológicos directamente sin necesidad de \textit{traducir} a ortográficos; por otro lado, facilita el tratamiento informático de los caracteres, pues cada unidad textual mínima está representada por un único carácter digital. El cambio supone una ventaja, en tanto que, al distinguir entre vocales y vocoides, desaparecen las ambigüedades. Asimismo, se presta mejor a las generalizaciones que el análisis grafémico —por ejemplo, tratamos diptongos y sinalefas como uniones equivalentes de un núcleo silábico y una vocal no silábica, y no requiere prever excepciones en los segmentos consonánticos para tratar los dígrafos—. Por otra parte, permite emplear más de un acento prosódico, como es menester con los adverbios en \textit{-mente}\index{adverbio!{en -mente}@{en -\textit{mente}}}.

Otra mejora sustancial deriva del procedimiento para hacer la división silábica. Aunque la ortografía y la prosodia suelen ir de la mano las más de las veces, sus caminos se separan ocasionalmente. Si bien esto encuentra una solución editorial mediante el empleo de diacríticos (\textit{crüel}), este no es siempre el caso y, de cualquier modo, no lo es para formas verbales excepcionales de uso común. El poeta escribe sus versos en el papel y no al contrario, por lo  que poco sentido tiene ceñirse a la ortografía cuando esta contraviene la prosodia.

El complemento sintáctico al análisis morfológico supone otro avance. Numerales, nombres propios compuestos, fórmulas de tratamiento... Aunque resulten excepcionales en la lírica, son corrientes en el teatro. Por eso, ignorarlos no provocaría fallos marginales sin apenas efecto en el conjunto, sino errores recurrentes acumulados que desvirtuarían los metadatos del corpus anotado. Por lo tanto, su consideración resulta crucial para obtener resultados aceptables.

En cuanto al sistema de reglas de reajuste del verso, los resultados sugieren que el propuesto da un paso en la buena dirección. El modelo de prioridades combinado con la reevaluación en tiempo real alcanza una precisión que supera todos los sistemas de escansión del español publicados. Las ventajas del tratamiento recursivo por prioridades frente al análisis lineal del verso son evidentes: una vez que hemos hecho un reajuste versal, la configuración silábica es otra, por lo que resulta incoherente seguir analizándolo como si de la secuencia de sonidos inicial se tratase.

Una vez tenemos los versos con sus correspondientes metros, podemos entenderlos como una lista de elementos en los que buscar patrones. Estos responden a agrupaciones según el número de sílabas de los versos, por un lado, y, por otro —de vital importancia—, de acuerdo con su rima y el tipo de esta.

Por otra parte, el texto dramático, a diferencia del lírico, es polimétrico. Los distintos segmentos se suceden sin solución de continuidad, por lo que resulta imprescindible establecer sus límites para caracterizarlos. Al contrario que con la escansión de poemas\index{poema}, donde el principio y el fin de la unidad vienen dados, las agrupaciones de versos del texto dramático no presentan una separación explícita entre ellas. Por lo tanto, antes de clasificar, hay que segmentar el texto según las características comunes de los versos.

Los versos se agrupan en virtud de su caracterización métrica y de rima, conociendo que la mayoría de las estrofas\index{estrofa} son consonantes y las asonancias aparecen típicamente en romances, siendo las estrofas asonantadas excepcionales. En consecuencia, buscamos esquemas rítmicos consonantes que repitan el patrón de estrofas del mayor número de versos posible y, de no encontrarlas, probamos en orden sucesivo con las estrofas inmediatamente menores según el número de versos. Los versos consonantes se clasifican de este modo en estrofas simples. A continuación, evaluamos  los versos que han quedado sin marcar. Primero cotejamos los metros asonantados con un número de versos fijo, tal como, por ejemplo, la cuarteta tirana y, después, determinamos los de longitud variable, como los romances. Una vez hechas las agrupaciones simples, se intenta unirlas en otras complejas donde sea posible. Esto es, el comienzo de \textit{La vida es sueño} daría como resultado del paso anterior una tirada  de cincuenta y un pareados de versos heptasílabos y endecasílabos. Tras esta segunda evaluación, la serie se agruparía como una única silva.

De aceptar cierto margen de error en la agrupación para el tamaño del verso, considerando la posibilidad de una escansión en falso —que, si bien infrecuente, no es imposible, dada nuestra aproximación a la polimetría—, puede aprovecharse el resultado para corregir posibles errores. Esto es, la rima ofrece la guía que seguir y el número de sílabas se considera una aproximación. Una vez agrupadas las estrofas\index{estrofa}, se deduce el número correcto de sílabas, tanto por las reglas poéticas como, mediante inducción, a partir de versos sin ambigüedades de la misma unidad. Con esto, pueden volver a escandirse aquellos que presenten un metro inusual, pero, en esta ocasión, forzando el número de sílabas en la escansión según el contexto.

\subsection{III. ¿Cómo organizar la información resultante en un corpus anotado?}

La última cuestión, si bien es tal vez la más sencilla de resolver en cuanto a la técnica, presenta mayor complejidad desde el punto de vista conceptual. No se trata solo de identificar entidades, sino de caracterizarlas de tal manera que su representación no dé lugar a ambigüedades. El primer paso era convertir las ediciones digitales a archivos de texto, eliminando todo el contenido extratextual. Si bien esto permite hacer búsquedas de palabras de forma relativamente sencilla, los resultados son poco prácticos, en tanto que ubicarlos requiere valorar manualmente su posición dentro del texto. Peor aún, al haber limpiado la fuente de todo aquello ajeno al propio texto, se pierden datos como, por ejemplo, la paginación. Por lo tanto, esta primera aproximación valdría para identificar los textos que contienen ocurrencias de una búsqueda dada y hacer una estimación a ojo de su posición, con la que buscarla de vuelta en el texto de origen. Esto, ya de por sí, resulta de gran utilidad, pues agiliza la investigación cualitativa sobremanera. Permitiría incluso la cuantitativa, considerando, por ejemplo, el número de ocurrencias de una búsqueda, lo que se extendería a análisis más detallados, desde la complejidad léxica del texto a la estilometría\index{estilometría}. Sin embargo, detenerse aquí dejaría sin explotar todo el potencial que ofrecen las nuevas tecnologías.

Lo siguiente sería ir un paso más allá de las palabras y considerar sus relaciones. El ejemplo por antonomasia en el mundo de la edición digital es el estándar \ac{xmltei}. Este caracteriza los elementos textuales mediante una jerarquía, por lo que un verso pertenece a un grupo de versos\index{verso}, estos, junto a la didascalia de personaje, el identificador del locutor\index{locutor} y otros datos, a un parlamento, el parlamento\index{parlamento} a una escena, la escena\index{escena} a un acto\index{acto}, y el acto al cuerpo del texto. El problema que presenta es que requiere categorizar y ordenar los elementos explícitamente, por lo que se necesitaría procesar el texto de origen —mejor dicho, los textos, cientos, tal vez miles, de comedias, literalmente millones de versos— manualmente. La inviabilidad de esto es evidente, por lo que proponemos un pequeño rodeo. Se trata de preparar los archivos de texto de tal manera que, sin alterar en lo esencial el método ordinario, se reflejen los elementos textuales sin ambigüedades y, hasta cierto punto, de forma implícita, al imitar las convenciones editoriales de maquetación del drama. Así, es el propio texto de origen el que proporciona la información requerida. Sin que haya una diferencia notable en las posibilidades de uso finalista con respecto a los archivos de texto ordinarios, ahora es posible manipular y analizar la información dramática automáticamente y clasificar las entidades que contiene, lo que sí es un paso cualitativo sustancial.

Aunque los datos semiestructurados resultan convenientes para almacenar la información de partida, la metainformación y sus relaciones, no se prestan bien a tareas que demandan un trabajo intensivo de búsqueda o el cotejo masivo de datos. Precisamente, ese tipo de trabajos son los que requieren los análisis estadísticos o las bases de datos, por lo que ambos trabajan con estructuras tabulares. En consecuencia, se ha de traducir la disposición semiestructurada a una tabla secuencial, en la que las relaciones entre entidades no se representen a través de la jerarquía, sino mediante los valores de sus columnas.

Nuestra aproximación consiste en tomar aquí como unidad mínima no el parlamento\index{parlamento} o las líneas, sino el verso y, a idéntico nivel jerárquico, las acotaciones y las líneas en prosa y ecos, que se considerarían \textit{versos} carentes de metro, así como locutor y parlamento, según el caso, que se numerarían aparte. A cada una de estas unidades se le asigna una fila, que se caracteriza en las columnas mediante variables cuantitativas y categóricas. De esta manera, unas columnas representan la información metatextual, como el autor, título o datos de la edición; otras columnas contienen información estructural, como el número de acto, de parlamento (o acotación) y de verso; otras, información dramática, como las líneas y el personaje o personajes que la pronuncian y, por último, aquellas que proporcionan información métrica, como el ritmo\index{ritmo}, metro, rima\index{rima}, estrofa\index{estrofa} a la que pertenece y el tipo de esta. La realización física de la estructura es un archivo \ac{csv}, ya que este formato es compatible tanto con los paquetes de estadística como las bases de datos más comunes.

\subsection{Poniendo todo junto}
En resumen, la cadena de trabajo que proponemos comienza con la corrección semimanual de cada uno de los textos de origen, con la peculiaridad de que el resultado ha de facilitar la distinción entre entidades textuales. Esto permite identificar los versos y su contexto para crear una tabla a partir de ellos. Una de estas columnas contiene el verso en cuestión, que puede analizarse mediante \ac{pln} para añadir la información métrica en nuevas columnas.

Esta escansión se lleva a cabo mediante el análisis morfosintáctico para identificar los acentos prosódicos del verso. Después, se corrige el metro mediante la aplicación de las licencias poéticas. Este proceso se basa en asignar un grado de preferencia a los contactos vocálicos y resolver potenciales uniones o separaciones recursivamente. La preferencia se confiere según criterios ortológicos. La resolución consiste en el ajuste versal atendiendo al contacto vocálico de preferencia más alta, tras lo que, si el metro sigue sin cuadrar, vuelve a evaluarse el verso ajustado, y se repite el proceso hasta obtener el metro esperado; si no es posible, el proceso comienza de nuevo desde el principio pero con otro metro objetivo. Los versos examinados se agrupan finalmente atendiendo a la información obtenida. La tabla resultante de procesar un número suficiente de obras —casi medio millar en el proyecto Sound and Meaning a la fecha de redactar estas últimas líneas—, permite realizar análisis de corpus en función de todas las variables reflejadas en ella.

\section{Cuestiones abiertas y unas palabras finales}
Durante el transcurso de este trabajo han surgido ciertas cuestiones, unas a raíz de problemas a resolver, otras como producto inesperado de los experimentos. Algunas de ellas tienen cierto interés por sí mismas, por lo que no podíamos poner fin a este trabajo sin notar los interrogantes que abren para —esperemos— ser explorados en  el futuro como merecen. Mencionaremos las tres que más nos han llamado la atención.

Primero, durante las largas cavilaciones para abordar la resolución de las sinalefas, se hizo necesario considerar también aquellas que ocurren en versos compartidos. Como resulta evidente, no se da un desplazamiento articulatorio entre dos cavidades bucales, por lo que la situación ha de resolverse de otra manera. Esto condujo a la realización de análisis fonéticos acústicos, que revelaron diversas técnicas escénicas para enfrentarse a la peliaguda situación. Entre los hallazgos más interesantes destaca que algunas de estas estrategias entraban en abierta contradicción con la forma poética del verso, al hacer alteraciones espurias en el recuento silábico. Esto aconseja recopilar un número mayor de muestras para poder llevar a cabo un estudio descriptivo de las prácticas teatrales al respecto. Asimismo, da pie a reflexionar sobre la aproximación al fenómeno más apropiada, respetuosa con el espíritu de la obra y acorde a la sensibilidad del poeta.

En segundo lugar, al examinar los textos, hemos encontrado otro fenómeno que se observa  muy especialmente también en los versos compartidos. Se trata de la presencia de acentos contiguos. Si bien en la lírica estos se prestan mejor o peor a ser reducidos donde convenga según las necesidades del metro y el gusto del poeta, este no es siempre el caso del texto dramático. En una serie de réplicas y contrarréplicas monosilábicas, por ejemplo, resulta no únicamente difícil privar de la calidad tónica a las líneas allí donde lo requiera el metro, sino que, de hacerlo, se correría el riesgo de socavar el efecto dramático de los parlamentos. Resulta difícil reconciliar métrica y acción en estos casos porque demandan acciones contradictorias. ¿Cuál de las dos ha de satisfacer la resolución del metro? Cabría asimismo plantearse la cuestión desde el punto de vista compositivo, preguntarse qué idea podría tener el autor cuando pergeñó el verso.

Tercero y último, durante las pruebas, aquellas obras que presentaban errores de edición, transmisión o composición que afectaban a los versos producían inconsistencias fácilmente visibles en los resultados, por lo que se planteó la posibilidad de crear un paquete dotado de una interfaz de usuario sencilla que permitiera a cualquiera, aun sin conocimientos técnicos, llevar a cabo análisis automáticos y cotejar los resultados. Esto permitiría en cuestión de segundos encontrar metros y rimas inconsistentes, lo que podría agilizar la edición de textos, al menos en lo métrico, cuando la ocasional palabra mal puesta o enmendada altera el recuento silábico o el esquema de la rima.

Arribamos por fin a puerto. Atrás quedan las ideas y conceptos que hemos discutido y llevado a la práctica para marcar el rumbo que trazaron los interrogantes formulados al zarpar. Durante esta travesía, hemos ido esbozando propuestas en cada singladura, las hemos refinado en la siguiente y, por último, las hemos compendiado ya desbastadas en estas páginas postreras. Ha sido un largo y arduo viaje, este que ahora acabamos. Concluimos con la esperanza de que, más allá del mucho o poco interés que este trabajo pueda suscitarle al lector —que de más ha hecho llegando hasta aquí—, le sirva al menos como acicate para volver a explorar el teatro clásico español con la idea de contemplarlo desde una perspectiva diferente, nueva o antigua.
