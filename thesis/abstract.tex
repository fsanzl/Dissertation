\invisiblesection{Zusammenfassung auf Deutsch}
\vspace*{\fill}\begin{german}\begin{abstract}
		Ziel dieser Doktorarbeit ist es, der spanischen Philologie neue Methoden, Werkzeuge und Proofs of Concept zur Verfügung zu stellen, welche die digital-humanistischen Studien des spanischen Theaters der Frühneuzeit  erleichtern. Insbesondere entwickelt es Analysemethoden für polymetrisch versifizierte dramatische Werke des Goldenen Zeitalters, ausgehend von einer theoretischen Basis dramatologischer, poetischer und phonologischer Elemente, um diese als Algorithmen und Datenstrukturen zu formalisieren, die für die Implementierung in Computerprogramme geeignet sind. Es greift weniger auf Digital Humanities als methodischen Rahmen zurück, sondern sicht diese als Empfänger der Ergebnisse. Da die Eignung der vorgeschlagenen Formalisierung getestet wird, indem sie in reale Programme übersetzt wird, berücksichtigt das Design Programmiertechniken sowie die Verarbeitung natürlicher Sprache und Computerlinguistik. Der in dieser Arbeit vorgestellte Ansatz zielt darauf ab, die Sammlung und Klassifizierung von Daten zu erleichtern. Erstens schlägt es Ontologien und Klassen von Textelementen und ihre Beziehungen in Stücken des spanischen Goldenen Zeitalters vor; zweitens beschreibt es automatische Verarbeitungsmethoden, um sie zu extrahieren; schließlich wird untersucht, wie die resultierenden Informationen in Datenstrukturen organisiert werden können, welche für die weitere digitale Analyse geeignet sind.
		
		\smallskip
		\noindent\textbf{Schlagwörter:} literarische Analyse, Fernlesen, Digital Humanities, digitale Philologie, Spanisches Goldenes Zeitalter, Natural Language Processing, automatische Skansion
\end{abstract}\end{german}\vspace*{\fill}\newpage
\invisiblesection{Abstract}
\vspace*{\fill}\begin{english}\begin{abstract}
		This dissertation aims at equipping Spanish Philology with novel methods, tools, and proofs of concept to facilitate Digital Humanities studies of Spanish plays. In particular, it develops methods to analyse Spanish Golden Age versified plays. It departs from the theoretical grounds of dramatology, poetry, and phonology to formalise its elements as algorithms and data structures suitable for implementation as software. It resorts to Digital Humanities not as much as a methodological framework but as a target audience for the outcomes. Since the formalisation's suitability is tested by translating it into actual computer programmes, the design considers programming techniques, natural language processing and computational linguistics. The approach presented in this work intends to expedite data retrieval and classification. Firstly, it proposes ontologies and classes of textual elements in Golden Age plays and their interrelations; secondly, it describes automatic processing methods to extract them; finally, it explores how to organise the resulting information into data structures suitable for further digital analysis.
		
		\smallskip
		\noindent\textbf{Keywords:} literary analyisis, distant reading, digital humanities, digital philology, Spanish Golden Age, Siglo de Oro, Natural Language Processing, metrical scansion
\end{abstract}\end{english}\vspace*{\fill}\newpage
\invisiblesection{Resumen en español}
\vspace*{\fill}
\begin{abstract}
Esta tesis doctoral aspira a proveer a la filología española de nuevos métodos, herramientas y pruebas de concepto que faciliten los estudios humanísticos digitales del teatro aurisecular. En concreto, desarrolla métodos de análisis de obras dramáticas versificadas polimétricas del Siglo de Oro. Parte de una base teórica de elementos dramatológicos, poéticos y fonológicos, para formalizar estos como algoritmos y estructuras de datos aptos para su implementación en programas informáticos. Contempla asimismo las humanidades digitales no tanto como marco metodológico sino como receptor de los resultados. En tanto que la idoneidad de la formalización propuesta se somete a prueba mediante su traducción a programas reales, el diseño considera técnicas de programación, procesamiento del lenguaje natural y lingüística computacional. La aproximación que presenta este trabajo pretende facilitar la recolección y clasificación de datos. Primero, propone ontologías y clases de características textuales y sus relaciones en las obras teatrales áureas; segundo, describe métodos de procesamiento automático para extraer dichos elementos; finalmente, explora cómo organizar la información resultante en estructuras de datos aptas para análisis digitales ulteriores.

\smallskip
\noindent\textbf{Palabras clave:} análisis literario, lectura distante, humanidades digitales, filología digital, Siglo de Oro, Procesamiento del Lenguaje Natural, escansión automática
\end{abstract}\vspace*{\fill}\newpage

