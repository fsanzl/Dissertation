\thispagestyle{empty}\newdimen\origiwspc%
\newdimen\origiwstr%
\origiwspc=\fontdimen2\font% original inter word space
\origiwstr=\fontdimen3\font% original inter word stretch

\begin{changemargin}{-13mm}{-8mm}{-30mm}
	\vspace*{-20.1mm}\begin{flushright}
		\includegraphics{images/university.png}
	\end{flushright}
	
	\strut\\[8.5mm]
	\begin{center}
		{\textbf{\MakeUppercase{\huge\Arial Dissertation / Doctoral Thesis}}} \\[16.2mm]\strut
		{\normalsize\Arial{}Titel der Dissertation / Title of the Doctoral Thesis}\\[-0.9mm]				
		{\Large\Arial „Ritmo y estructura de la comedia áurea:\\posibilidades y límites del análisis digital automático del teatro español clásico“}\\[11.7mm]
		{\normalsize\Arial{}verfasst von / submitted by} \\[1.5mm]
		{\large\Arial{}Lic. Mag. Fernando Sanz-Lázaro} \\[24.2mm]
		{\normalsize\Arial angestrebter akademischer Grad / in partial fulfilment of the requirements for the degree of} \\
		\large{\Arial{}Doktor der Philosophie (Dr.phil.)}
	\end{center}\strut\\[1mm] \strut
	\begin{flushleft}
		\normalsize{\Arial{}Wien, 2023 / Vienna 2023} \\[9mm]
		\SingleSpacing
		\begin{tabular}{@{}ll}
			\small \parbox[t]{90mm}{\Arial{}Studienkennzahl lt.\ Studienblatt / \\
				degree programme code as it appears on the student\\
				record sheet:}
			&\strut \hspace{1cm}
			{\Arial\small A 792 236} \\[11mm]
			\small \parbox[t]{90mm}{\Arial{}Dissertationsgebiet  lt. Studienblatt / \\
				field of study as it appears on the student record sheet:}
			&\strut \hspace{1cm}
			{\Arial\small Romanistik / Romance Studies} \\[11mm]
			\small \parbox[t]{90mm}{\Arial{}Betreut von / Supervisor:}
			&\strut \hspace{1cm}
			{\Arial\small PD Mag. Dr. Wolfram Aichinger}
		\end{tabular}
	\end{flushleft}\vfill
	\strut
\end{changemargin}
\fontdimen2\font=\origiwspc% (original) inter word space
\fontdimen3\font=\origiwstr% (original) inter word stretch
\newpage
\thispagestyle{empty}
\begin{center}
	\fbox{\begin{minipage}{\linewidth}
			Esta tesis ha sido amparada por los proyectos de investigación \textit{El Calderón Cómico} (\href{https://pf.fwf.ac.at/en/research-in-practice/project-finder/37881}{FWF P 29115}) e \textit{Interpretation of Childbirth in Early Modern Spain} (\href{https://pf.fwf.ac.at/en/research-in-practice/project-finder/46258}{FWF  P 32263-G30}), dirigidos por el Dr. Wolfram Aichinger, y \textit{Sound and Meaning in Spanish Golden Age Literature} (\href{https://pf.fwf.ac.at/en/research-in-practice/project-finder/46968}{FWF P 32563}), dirigido por el Dr. Simon Kroll.
	\end{minipage}}
	\strut\vspace{60ex}
	
	\strut
	\begin{small}
	\begin{minipage}{\linewidth}
		Este trabajo se compuso con \XeLaTeX{} usando los tipos de letra CMU Serif y {\footnotesize\DVS{}DejaVu Sans} para el cuerpo del texto,  \texttt{CMU Typewriter} para los listados de código fuente y $Latin\ Modern\ Math$ para las fórmulas, todos bajo licencia libre; la portada incluye el tipo de letra comercial {\small\Arial{}Arial} de acuerdo con las directrices de la Universidad.
		\end{minipage}
	\vfill
		\textcopyright{} 2023 Fernando Sanz-Lázaro
		
		Bajo licencia Creative Commons Attribution 4.0 International (CC BY 4.0)\\
		
		Puede encontrar los términos de la licencia en:\\ \url{https://creativecommons.org/licenses/by/4.0/legalcode.es}
		
		\includegraphics{images/by.png}
		
		Impreso en Viena
	\end{small}\end{center}\newpage
