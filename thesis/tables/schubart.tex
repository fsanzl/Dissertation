\begin{table}[!ht]
	\centering
	\begin{tabular}{lccccccccc}
		\toprule
		&\rot{Estructura}&\rot{Patetismo}&\rot{Expresión dramática}&\rot{Caracterización}&\rot{Gusto}&\rot{Colorido}&\rot{Versificación}&\rot{Moral}&\rot{\textbf{Estimación final}} \\
		\midrule
		Ariosto&0&15&10&15&14&15&16&10&13\\
		Boileau&18&16&12&14&17&14&13&16&12\\
		Cervantes&17&17&15&17&12&16&—&16&14\\ 
		Corneille&15&16&16&16&16&14&12&16&14\\ 
		Dante&12&15&8&17&12&15&14&14&13\\ 
		Eurípides&15&16&14&17&13&14&—&15&12\\ 
		\textit{Homero}&18&17&18&15&16&16&18&17&18\\ 
		Horacio&12&12&10&16&17&17&16&14&13\\ 
		Lucrecio&14&5&—&17&17&14&16&0&10\\ 
		\textit{Milton}&17&15&15&17&18&18&17&18&17\\ 
		Molière&15&17&17&17&15&16&—&16&14\\ 
		Píndaro&10&10&—&17&17&16&—&17&13\\ 
		Pope&16&17&12&17&16&15&15&17&13\\ 
		Racine&17&16&15&15&17&13&12&15&13\\ 
		\textit{Shakespeare}&0&18&18&18&10&17&10&18&18\\ 
		Sófocles&18&16&15&15&16&14&—&16&13\\ 
		Spenser&8&15&10&16&17&17&17&17&14\\ 
		Tasso&17&14&14&13&12&13&16&13&12\\ 
		Terencio&18&12&10&12&17&14&—&16&10\\ 
		\textit{Virgilio}&17&16&10&17&18&17&17&17&16\\ 
		\bottomrule
		\multicolumn{10}{l}{\footnotesize Cursiva del original.}
	\end{tabular}
	\caption{Escala de los jóvenes poetas según Akenside.}
	\label{tab:schubart}
\end{table}
