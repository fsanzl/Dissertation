\begin{table}[!ht]
	\centering\scriptsize
\begin{tabular}{rp{5.6cm}rrlrl}
\toprule
\textbf{V} &\textbf{Texto} &\textbf{Síl.} & \textbf{Núcl.} & \textbf{Ason.} & \textbf{Cons.} & \textbf{Ritmo}\\
\midrule
-1 & Sale en lo alto de un monte \textsc{Rosaura} en hábito de hombre,de camino, y en representando los primeros versos va bajando.&&&&&\\
1 & Hipogrifo violento, & 7 & ioIooEo & eo & ento & \texttt{--+---+-}\\
2 & que corriste parejas con el viento, & 11 &eoIeaEaoeEo & eo &ento & \texttt{--+--+---+-}\\
3 & ¿dónde rayo sin llama, & 7 &OeAoiAa & aa & ama & \texttt{+-+--+-}\\
4 & pájaro sin matiz, pez sin escama & 11 &AaoiaIEieAa & aa & ama & \texttt{+----++--+-}\\
5 &  y bruto sin instinto & 7 &iUoiiIo & io & into & \texttt{-+---+-}\\
6 & natural, al confuso laberinto & 11 &auAaoUoaeIo & io & into & \texttt{--+--+---+-}\\
7 &  de esas desnudas peñas & 7 &EaeUaEa & ea & eɲas & \texttt{+--+-+-}\\
8 & te desbocas, te arrastras y despeñas? & 11 &eeOaaAaieEa & ea & eɲas & \texttt{--+--+---+-}\\
9 & Quédate en este monte, & 7 &EaeEeOe & oe & onte & \texttt{+--+-+-}\\
10 &  donde tengan los brutos su Faetonte; & 11 &oeEaoUoueOe & oe & onte & \texttt{--+--+---+-}\\
11 &  que yo, sin más camino & 7 &eOiAaIo & io & ino & \texttt{-+-+-+-}\\
12 & que el que me dan las leyes del destino, & 11 &eeeAaEeeeIo & io & ino & \texttt{---+-+---+-}\\
13 & ciega y desesperada, & 7 &EieeeAa & aa & ada & \texttt{+----+-}\\
14 & bajaré la cabeza enmarañada & 11 &aaEaaEeaaAa & aa & ada & \texttt{--+--+---+-}\\
15 & deste monte eminente & 7 &EeOeiEe & ee & ente & \texttt{+-+--+-}\\
16 &  que arruga el sol el ceño de la frente. & 11 &aUeOeEoeaEe & ee & ente & \texttt{-+-+-+---+-}\\
17 &  Mal, Polonia, recibes & 7 &AoOaeIe & ie & ibes & \texttt{+-+--+-}\\
18 & a un extranjero, pues con sangre escribes & 11 &UeaEoeoAeIe & ie & ibes & \texttt{+--+---+-+-}\\
19 & su entrada en tus arenas; & 7 &eAeuaEa & ea & enas & \texttt{-+---+-}\\
20 &  y apenas llega, cuando llega a penas. & 11 &aEaEaaoEaEa & ea & enas & \texttt{-+-+---+-+-}\\
21 & Bien mi suerte lo dice; & 7 &EiEeoIe & ie & iθe & \texttt{+-+--+-}\\
22 & mas ¿dónde halló piedad un infelice? & 11 &aOaOeAUieIe & ie & iθe & \texttt{-+-+-++--+-}\\
-22 &(Sale \textsc{Clarín}, gracioso.)&&&&&\\
23 &  Di dos, y no me dejes & 7 &IOiOeEe & ee & exes & \texttt{++-+-+-}\\
24 & en la posada a mí cuando te quejes; & 11 &eaoAaIaoeEe & ee & exes & \texttt{---+-+---+-}\\
25 &  que si dos hemos sido & 7 &eiOEoIo & io & ido & \texttt{--++-+-}\\
26 &  los que de nuestra patria hemos salido & 11 &oeeeaAAoaIo & io & ido & \texttt{-----++--+-}\\
27 & a probar aventuras, & 7 &aoAaeUa & ua & uɾas & \texttt{--+--+-}\\
28 & dos los que entre desdichas y locuras & 11 &OoeeeIaioUa & ua & uɾas & \texttt{+----+---+-}\\
29 &  aquí habemos llegado, & 7 &aIEoeAo & ao & ado & \texttt{-++--+-}\\
30 &  y dos los que del monte hemos rodado, & 11 &iOoeeOEooAo & ao & ado & \texttt{-+---++--+-}\\
31 & ¿no es razón que yo sienta & 7 &OaOeOEa & ea & enta & \texttt{+-+-++-}\\
32 &  meterme en el pesar y no en la cuenta? & 11 &eEeeeAiOaEa & ea & enta & \texttt{-+---+-+-+-}\\
...&&&&&&\\
\bottomrule
\end{tabular}
 \caption{Tabla de atributos métricos.}
        \label{tab:dramatac}
\end{table}
