\begin{table}[!ht]
	\centering\scriptsize
\begin{tabular}{lcrrlcp{5.7cm}}
\toprule
\textbf{Título} & \textbf{Acto} & \textbf{Parl.} & \textbf{Verso} & \textbf{Pers.} & \textbf{Sexo} & \textbf{Texto} \\
\midrule
\textit{La vida es sueño}&I&-1&-1&\textit{Acotación}&& Sale en lo alto de un monte \textsc{Rosaura} en hábito de hombre,de camino, y en representando los primeros versos va bajando. \\
\textit{La vida es sueño}&I&1&1& \textsc{Rosaura} & {\DVS♀} & Hipogrifo violento, \\
\textit{La vida es sueño}&I&1&2& \textsc{Rosaura} &  {\DVS♀} & que corriste parejas con el viento, \\
\textit{La vida es sueño}&I&1&3& \textsc{Rosaura} &{\DVS♀} & ¿dónde rayo sin llama, \\
\textit{La vida es sueño}&I&1&4& \textsc{Rosaura} &  {\DVS♀} & pájaro sin matiz, pez sin escama\\
\textit{La vida es sueño}&I&1&5& \textsc{Rosaura} &  {\DVS♀} & y bruto sin instinto \\
\textit{La vida es sueño}&I&1&6& \textsc{Rosaura} &  {\DVS♀} & natural, al confuso laberinto \\
\textit{La vida es sueño}&I&1&7& \textsc{Rosaura} & {\DVS♀}& de esas desnudas peñas \\
\textit{La vida es sueño}&I&1&8& \textsc{Rosaura} & {\DVS♀} & te desbocas, te arrastras y despeñas? \\
\textit{La vida es sueño}&I&1&9& \textsc{Rosaura} &{\DVS♀} & Quédate en este monte, \\
\textit{La vida es sueño}&I&1&10 & \textsc{Rosaura}  & {\DVS♀} & donde tengan los brutos su Faetonte; \\
\textit{La vida es sueño}&I&1&11 & \textsc{Rosaura}  & {\DVS♀} & que yo, sin más camino \\
\textit{La vida es sueño}&I&1&12 & \textsc{Rosaura}  & {\DVS♀} & que el que me dan las leyes del destino, \\
\textit{La vida es sueño}&I&1&13 & \textsc{Rosaura} & {\DVS♀} & ciega y desesperada, \\
\textit{La vida es sueño}&I&1&14 & \textsc{Rosaura}  & {\DVS♀} & bajaré la cabeza enmarañada \\
\textit{La vida es sueño}&I&1&15 & \textsc{Rosaura} & {\DVS♀} & deste monte eminente \\
\textit{La vida es sueño}&I&1&16 & \textsc{Rosaura}  & {\DVS♀} & que arruga el sol el ceño de la frente. \\
\textit{La vida es sueño}&I&1&17 & \textsc{Rosaura} & {\DVS♀} & Mal, Polonia, recibes \\
\textit{La vida es sueño}&I&1&18 & \textsc{Rosaura}  & {\DVS♀} & a un extranjero, pues con sangre escribes \\
\textit{La vida es sueño}&I&1&19 & \textsc{Rosaura}  & {\DVS♀} & su entrada en tus arenas; \\
\textit{La vida es sueño}&I&1&20 & \textsc{Rosaura}  & {\DVS♀} & y apenas llega, cuando llega a penas. \\
\textit{La vida es sueño}&I&1&21 & \textsc{Rosaura}  &{\DVS♀} & Bien mi suerte lo dice; \\
\textit{La vida es sueño}&I&1&22 & \textsc{Rosaura}  & {\DVS♀} & mas ¿dónde halló piedad un infelice? \\
\textit{La vida es sueño}&I&-2&-22 & \textit{Acotación}   &  & (Sale \textsc{Clarín}, gracioso.) \\
\textit{La vida es sueño}&I&2&23 & \textsc{Clarín}  & {\DVS♂} & Di dos, y no me dejes \\
\textit{La vida es sueño}&I&2&24 & \textsc{Clarín}  & {\DVS♂} & en la posada a mí cuando te quejes; \\
\textit{La vida es sueño}&I&2&25 & \textsc{Clarín}  & {\DVS♂} & que si dos hemos sido \\
\textit{La vida es sueño}&I&2&26& \textsc{Clarín}  & {\DVS♂} & los que de nuestra patria hemos salido \\
\textit{La vida es sueño}&I&2&27& \textsc{Clarín}  & {\DVS♂} & a probar aventuras, \\
\textit{La vida es sueño}&I&2&28& \textsc{Clarín}  & {\DVS♂} & dos los que entre desdichas y locuras \\
\textit{La vida es sueño}&I&2&29& \textsc{Clarín}  & {\DVS♂} & aquí habemos llegado, \\
\textit{La vida es sueño}&I&2&30& \textsc{Clarín}  & {\DVS♂} & y dos los que del monte hemos rodado, \\
...&&&&&&\\
\bottomrule
\end{tabular}
  \caption{Tabla de atributos dramáticos.}
        \label{tab:dramatab}
\end{table}
