\chapter*{Agradecimientos}\markboth{Agradecimientos}{Agradecimientos}

Esta tesis ha sido posible gracias al concurso de un sinnúmero de personas, que, lo sepan o no, han puesto su granito de arena —paladas, alguno—. Sirvan estas líneas al menos como un mínimo reconocimiento a todos ellos, ya que jamás podré pagarles como verdaderamente merecen. 

Para comenzar, doy las gracias a Wolfram Aichinger, mi director de tesis, amigo y maestro, por no dejarme perder de vista lo que de verdad importa, empezando por el mismo título del trabajo y acabando con aquel curso de literatura que impartió durante el primer semestre de 2014 —y mío en la Universidad de Viena— en el Auditorio A de este viejo hospital general, donde entendí hacia dónde ir.

Vaya un agradecimiento muy especial para Simon Kroll por acogerme en su equipo, motivar esta disertación y tomarse un tiempo impagable en revisar el manuscrito con la minuciosidad del editor crítico que es, pero también por todo lo que he aprendido junto a él y por estar siempre alerta para señalar nuevos caminos por explorar.
 
Gracias a  Álvaro Cuéllar, por el torbellino de ideas que ha traído a Viena consigo, por redescubrir \textit{La francesa Laura} y porque, en el breve tiempo que hemos trabajado juntos, ha inspirado no pocas cosas y contribuido a mejorar muchas otras, incluyendo su lectura crítica del manuscrito. Hago una acotación aquí —\textit{pun intended}— para dar también las gracias a Clara Monzó, calderonista de pro y \textit{beta tester} sobrevenida. La tesis debe mucho a ambos.

Por supuesto, estoy obligado para con todos los investigadores cuyo trabajo ha sentado los cimientos de este —me alzo a hombros de gigantes—, algunos de los cuales he tenido el honor y el placer de conocer en persona a lo largo de estos años: Hanno Ehrlicher, Antonio Sánchez Jiménez y Ramón Valdés Gázquez, por su amabilidad y, sobre todo, por las ideas que he tomado prestadas de los trabajos de cada uno; Markus Ebenhoch, por su lectura y comentarios al primer capítulo teórico; Alejandro García Reidy, por sus detallados reportes de error (¡con solución incluida!); Fernando Rodríguez-Gallego, por todas sus conversaciones con Simon que han terminado en mi código; Jos Sagüés †2014, por sacarme del aula a las tablas; María Victoria Curto, Isabel García Adánez, Gaston Gilabert, Laura Hernández-Lorenzo, Achim Hermann Hölter, Jörg Lehmann, Borja Navarro-Colorado, Mario Rossi, Christoph Strosetzki y Álvaro Torrente.

Los proyectos Sound and Meaning in Spanish Golden Age Literature,  Interpretation of Childbirth in Early Modern Spain y El Calderón Cómico han contribuido a que pudiera dedicarme a tiempo completo a menesteres académicos hasta el término de esta tesis. No ha de faltar tampoco mi reconocimiento a mis compañeros en ellos y del departamento: \begin{german}Sabrina Grohsebner, Hannah Mühlparzer, Tamara Hanus, Sara Brankovic, Hannah Fischer, Kurt Kriz, Christian Standhartinger, Marie Radinger, Marie-Louise Fürsinn, Marie Stockinger\end{german}, Emma Marcos, así como a todos aquellos con los que he tenido el gusto de trabajar y me han aportado nuevas cosas que llevar en mi bagaje. Mi deuda se extiende, claro está, al resto del Institut für Romanistik de la Universidad de Viena, que me ha visto crecer como hispanista.

Debo mi más profunda gratitud a Martina por todos sus sacrificios, sin los cuales no hubiera podido dedicar las inacabables horas que ha demandado esta empresa. Estoy asimismo obligado con la inquisitiva Carlotta y el siempre risueño Mateo, por todas los ratos con papá que les ha robado este trabajo. Vosotros tres dais sentido a todo. Es menester agradecer también a Vidal y Sonia su disposición para conseguirme rápidamente un sinfín de materiales a los que, de otra manera, quién sabe si habría echado el guante, además de la caza de erratas. Mención especial para Maricruz †2023, pues en aquellos libros gordos de Petete me topé inadvertidamente con la historia y los clásicos, sin los que hoy posiblemente no estaría escribiendo esto; vives en nuestra memoria. Gracias a Erika y Georg por su solicitud para echar siempre una mano con lo que sea. Y un recuerdo para los {M.M.R.}.

A todos los que se me han pasado, que sois legión, también estoy en deuda con vosotros.\cleardoublepage
