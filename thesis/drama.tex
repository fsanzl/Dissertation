%\qsetcnfont{helvetica}
%\begin{savequote}[65mm]\sffamily 
%	Prodigios verán los hombres\\
%	en tres actos, y ninguno\\
%	a su representación\\
%	faltará por mi descuido.\qauthor{Calderón, \textit{El gran teatro del mundo}}
%\end{savequote}
\chapter{Drama}
\epigraphhead[50]{\epigraph{Prodigios verán los hombres\\en tres actos, y ninguno\\a su representación\\faltará por mi descuido.}{Calderón, \textit{El gran teatro del mundo}}}

\section{Literatura escenificada}
El teatro es ilusión. Personas de carne y hueso \textit{interpretan} papeles y, al hacerlo, encarnan \textemdash en el sentido literal de la palabra\textemdash{} personajes\index{personaje} que no existían más que sobre el papel hasta que se abre el telón y se crea el mundo ilusorio del teatro, cuya efímera tangibilidad se esfuma en la nada en el mismo momento que cae por última vez la cortina. El teatro, según Aichinger~\parencite*[99-100]{aichinger2013b}, posee un carácter mágico que propicia una transformación vital de intensidad extraordinaria provocada por la literatura. Nos hace experimentar una versión de la vida aumentada, hasta el punto que olvidamos estar en un patio de butacas y, durante el tiempo que dura la representación, aceptamos como auténtico cuanto ocurre sobre las tablas. Olvidamos que estamos ante una realidad escenificada; no solo eso, sino que \textit{deseamos} olvidarlo; lo hacemos a sabiendas y de buena gana. El teatro no es un engaño al uso porque el espectador es consciente en todo momento de que la \textit{realidad} que se le está ofreciendo deja de serlo al terminar la función; incluso el montaje más realista rebosa artificio\footnote{En el teatro del {S}iglo de {O}ro, los costes de la puesta en escena de autos sacramentales se disparó entre 1601 y 1631 a causa de una espectacularización del evento \parencite[201]{ehrlicher2008}.}. No obstante, el espectador está más que dispuesto a dejar de lado su escepticismo, acepta contemplar esa nueva realidad en sus propios términos. Esta suspensión de la incredulidad es necesario para alcanzar la catarsis aristotélica y es, por lo tanto, un elemento esencial del teatro. No obstante, dejemos aquí la sala de la recepción literaria para, valga la metáfora, echar un vistazo al género dramático desde el foro.

Una característica definitoria del drama, aquella que lo diferencia de otros géneros literarios mayores, ya la señaló Aristóteles en su \textit{Poética} al conceder que, de la tragedia —\nolinebreak podríamos decir que por extensión también de otras formas dramáticas\nolinebreak—\nolinebreak, «el más importante de estos elementos es la estructuración de los hechos; porque la tragedia es imitación, no de personas, sino de una acción y de una vida, y la felicidad y la infelicidad están en la acción, y el fin es una acción, no una cualidad» (\poetics{1450a15-18}). No solo es crucial, nos atrevemos a decir, sino que esta estructura de los hechos constituye el drama mismo, pues su esencia última es la acción de los personajes. Al contrario que la narrativa y, aún más, la lírica, donde las descripciones, pensamientos o reflexiones reclaman para sí una función articuladora en la estructura interna de la obra, el drama se fundamenta en acciones.

Tal vez lo exponga Platón con más claridad cuando dice que la tragedia surge al retirar las palabras del poeta de la composición narrada, para dejar sin otro acompañamiento las intervenciones de los personajes\index{personaje} que en ella toman parte (\politeia{393a-394c}), esto es, los parlamentos\index{parlamento}. Dicho de otra manera, la fuerza motriz de la historia está en los diálogos de los personajes y no en la voz narrativa explícita del poeta. No quiere decir esto que la voz del narrador desaparezca por completo. Recordemos, por ejemplo, el comienzo de la película \textit{Bande à part}. El actor\index{actor} Sami Frey va conduciendo un descapotable por una carretera a las afueras de París. A su lado está sentado Claude Brasseur. Ambos conversan sobre un asunto que se traen entre manos, posiblemente un negocio algo turbio que entraña cierto riesgo. Ven a una chica conocida sobre una bicicleta. En un momento de silencio, se oye una voz superpuesta que dice algo como: «Mi historia comienza aquí, dos semanas después de conocer a Odile, Franz lleva en su coche a Arthur para enseñarle la casa». Tenemos una voz extradiegética que narra, en efecto, pero, su intervención no determina que estemos viendo a Franz llevando a Arthur a la casa. Ambos reconocen a Odile antes de la intervención del narrador. La película ha avanzado sin necesidad de este y sigue avanzando con independencia de su participación; el narrador no puede sino limitarse a glosar la acción que están llevando a cabo los personajes. En la obra narrativa, por el contrario, la voz narratorial tiene la facultad de establecer los hechos\footnote{No entraremos en si debemos fiarnos del narrador para no complicar innecesariamente el ejemplo. Sobre esto en el Siglo de Oro, ver Ehrlicher~\parencite*{ehrlicher2021}.} sin necesidad de la intervención directa de los personajes.

El poeta —\nolinebreak el buen poeta\nolinebreak— se retira además estilísticamente, porque, si se aviene a las exigencias del decoro, ha de hacer hablar a sus personajes\index{personaje} en observancia de lo que su naturaleza ficcional demanda: 
\blockquote{La avenencia del estilo y el carácter o sujeto interno se conoce como decoro o adecuación del estilo al contenido. El decoro es, en general, la \textit{voz ética} del poeta, la adaptación de su propia voz a la voz del personaje o al tono vocal que demanda el sujeto o el ambiente. Y, al igual que el estilo está en su más pura forma en la prosa, está el decoro en el drama en su más pura forma, donde el poeta no aparece en persona. El drama se describe, desde nuestro punto de vista actual, como  \textit{epos} o ficción absorbido por el decoro. \parencite[pp. 268-269; traducción propia]{frye1971}}

Esta predominancia de las acciones de los personajes\index{acción dramática} tiene una implicación adicional: la historia avanza más o menos en tiempo real a medida que se suceden las intervenciones. Al contrario que en una obra narrativa, en la que el tiempo se ralentiza o acelera a voluntad, la obra teatral requiere sincronía. Podemos detener el tiempo mientras Segismundo declama el celebérrimo soliloquio de \textit{La vida es sueño}, ¿pero qué ocurriría si un hipotético personaje entrara a interrumpir las disquisiciones metafísicas del prisionero? La mínima interacción nos hubiera obligado a poner a correr nuevamente el reloj. En este sentido, tal vez sean los soliloquios\index{soliloquio} los parlamentos\index{parlamento} menos dramáticos, los más cercanos a la lírica, en tanto que expresan la interioridad subjetiva de un narrador, «la pronunciación monológica de un \textit{yo}» \parencite[p. 191; traducción propia, énfasis añadido]{kayser1992}. Una buena porción de los apartes se interpreta de una forma similar, como una breve interrupción del curso de la historia. Sin embargo, en cualquier caso, no son estos recursos definitorios del género, ya que este acepta la creación sin recurrir a ellos. Por lo tanto, hemos de entenderlos más como una contingencia que como necesidad.

 Tampoco se le escapó esto a Hegel, quien, más de dos milenios después, afirma que \blockquote{la necesidad del drama es, en efecto, representar las acciones y relaciones humanas en el tiempo presente para la conciencia representativa y, por consiguiente,  manifestaciones lingüísticas de personas que representan la acción} \parencite*[3, C, III, \textsc{i}, a; traducción propia]{hegel2016}. Aquí tenemos un matiz fundamental, la \textit{manifestación lingüística}\footnote{\flushbottom\textit{Sprachliche Äußerung} en el original.} como acción. En efecto, en las obras teatrales aparecen acotaciones más o menos detalladas, aunque, incluso el dramaturgo más minucioso deja en gran medida la proxémica a un elemento mediador entre el texto y el espectador, normalmente el actor\index{actor}\footnote{Este elemento puede ser un sistema complejo, por ejemplo, el que forman el director de escena y el actor.}. Sin embargo, en condiciones normales, la esencia textual de la acción del personaje\index{acción dramática} llega al receptor sin haber sido alterada, pues el actor transmite el texto, si bien añadiendo matices propios. En otras palabras, el dramaturgo proporciona el armazón lingüístico sobre el que el actor construye la edificación interpretativa, donde emplaza las nuevas piezas sin modificar el andamiaje. Se trata sin duda de un ejercicio de transmedialidad \parencite[181-182]{tuerschmann2013}, en el que el texto se combina con la de la actuación en un flujo intermedial del que surge la totalidad del hecho dramático.  

No importa si presenciamos la representación de una tragedia de Eurípides o de una novísima pieza experimental, la naturaleza del texto como un discurso en tiempo real mediado por el actor\index{actor} no varía en lo concerniente al espectador. Esta y no otra es la naturaleza de la obra teatral. Así debemos abordarla si aspiramos a analizarla en sus propios términos.

\section{Aproximaciones al drama}
El drama tiene una naturaleza dual, ya que se observa desde una perspectiva escénica o una textual, sin perjuicio de que el texto sea parte constituyente de la puesta en escena. La disputa se remonta, al menos, a la polémica que sostuvieron los miembros del Círculo de Praga, Zich y Veltrusky. El primero defendía una tesis \textit{escenocéntrica}, que la obra dramática solo existía en tanto que se representa, mientras que el segundo contraargumentaba desde una posición \textit{textocéntrica}, argüía que el texto es una obra artística completa en sí misma que, además, determina el resultado de la representación \parencite[612-613]{schaeffer1995}. Esta diferencia ya estaba presente en el Siglo de Oro y poetas de la talla de Calderón así lo hicieron constar \parencite[122]{ehrlicher2023}.

\citeauthor{schaeffer1995}~\parencite*[614]{schaeffer1995} señala que el teatro se presta al estudio desde una perspectiva antropológica, semiótica, de análisis del discurso y mimética. El primer enfoque retoma la hipótesis aristotélica del origen religioso del drama. El problema que presenta esta aproximación es la falta de evidencia documental o arqueológica que la avale, por lo que hoy se tiende a considerar la representación ritual más como una de las múltiples formas que adopta el teatro que como su germen de las artes escénicas.

La aproximación semiótica \parencite[615-616]{schaeffer1995} toma el teatro como un polisistema compuesto de un conjunto de sistemas de signos visuales, verbales y sonoros. Este planteamiento presenta la dificultad de tener que segmentar el sistema en unidades de significado mínimas para establecer un \textit{código}. El teatro occidental —\nolinebreak al contrario que, por ejemplo, el \textit{nô} japonés\nolinebreak—\nolinebreak, carece de un sistema de unidades proxémicas aisladas con significado. Resulta llamativo que otro aspecto aparentemente asequible a la segmentación, las interacciones entre diálogo y música, se haya desarrollado tan poco —\nolinebreak tal vez por el carácter interdisciplinar del fenómeno, que requeriría aunar la crítica teatral y la musicología\nolinebreak—\nolinebreak. Entendido como texto, la obra teatral escapa a los principios del análisis narratológico, pues no cuenta una historia, sino que es una progresión dinámica de interacciones lingüísticas. En este sentido, resulta mucho más adecuada la segmentación en unidades deícticas referidas a los actantes que el análisis narratológico.

Con todo, el aspecto lingüístico está en el mismo centro del teatro, textual o escenificado.  Este lenguaje, además, no es el natural, sino su representación artística: está guiado por una aspiración estilística y responde a las necesidades comunicativas teatrales y no a las de los actos de comunicación corrientes \parencite[617]{schaeffer1995}. Hablamos, pues, de una \textit{doble enunciación}\footnote{\textit{Double énonciation} en francés.} en los actos de habla teatrales, ya que los emite el personaje y están dirigidos a su interlocutor en escena, pero, por otra parte, son también el discurso autorial dirigido al espectador.
\blockquote{Todo discurso en el teatro tiene dos emisores en la enunciación, el personaje y el escritor (como tiene dos receptores, otro personaje y el público\index{público}).  Esta ley del doble emisor de la enunciación es un elemento capital del texto teatral: es aquí donde se encuentra la falla inevitable que separa al personaje de su discurso y le impide constituirse como verdadero emisor de sus palabras. Cada vez que habla un personaje\index{personaje}, no lo hace en solitario, y el poeta habla al mismo tiempo por su boca; de aquí el dialogismo constitutivo del texto teatral. \parencite[p. 105; traducción propia]{ubersfeld1996}}

Al contrario que, por ejemplo, una homilía —teatralizada y sujeta a códigos escénicos convencionales \parencite[318-319]{ebenhoch2015}—, el teatro es ajeno a la transmisión inmediata. Si el dramaturgo interpreta sus propias líneas, actúa, representa necesariamente el papel de un personaje, mientras que el predicador no encuentra impedimento para hablar con su propia voz.

El análisis de la pieza teatral como un texto destinado a la representación propone entenderlo según cómo actúa en la audiencia mediante su escenificación. Los diálogos no se abordan desde la crítica literaria ni el análisis del discurso de una conversación no ficcional, sino como medio dramatúrgico. Se considera la tensión de un texto que no únicamente debe ser recitado sino representado dentro del contexto dramático.

Finalmente, considerando la función mimética del texto, esto es, desde la poética clásica, se observan dos aspectos principales: el estudio del conflicto y el del discurso dramático. En cuanto a lo primero, los personajes se distinguen de sus funciones dramáticas. Consideremos, por ejemplo, los movimientos de los \textit{actantes}, cada uno de los cuales posee un dominio constituido sintácticamente por los movimientos que le pertenecen y semánticamente por las máximas de acción pertinentes en su dominio sintáctico \parencite[618-620]{schaeffer1995}. El segundo aspecto lo propone \citeauthor{garcia1991}, quien lo denomina \textit{dramatología}, inspirado por la \textit{narratología} de Genette; lo desarrollaremos en la siguiente sección.

\section{Dramatología}
El concepto de \textit{dramatología}\index{dramatología} es una propuesta de \citeauthor{garcia1991} como alternativa a lo que él denomina pensamiento «débil», como una reacción a la terminología deliberadamente vaga que conduce a una lógica deficiente cuando esta se emplea para enlazar proposiciones \parencite[40]{garcia2020}. Trataremos de condensar aún más en esta sección lo que \citeauthor{garcia2020} ya sintetiza en \textit{Anatomía del drama}, bajo el pertinente subtítulo \textit{Una teoría fuerte del teatro}.

El primer presupuesto de la dramatología tiene raíces aristotélicas, pues toma prestados los dos modos de mímesis descritos en la \textit{Poética}:
\blockquote{En efecto, con los mismos medios es posible imitar las mismas cosas unas veces narrándolas (ya convirtiéndose hasta cierto punto en otro, como hace Homero, ya como uno mismo y sin cambiar), o bien presentando a todos los imitados como operantes y actuantes. (\poetics{1448a19-24})}

Sin embargo, \citeauthor{garcia2020} matiza que, cuando Aristóteles describió los modos miméticos, no había forma posible de conocer que, un par de milenios más tarde, de manera simultánea, unos hermanos de Besançon, por un lado, y un inventor estadounidense por otro, construirían sendos ingenios para proyectar imágenes en movimiento. El \textit{medio} de imitación y no el \textit{modo} es lo que crea la analogía del cine con el teatro y lo distingue de la narración. En efecto, si consideramos el modo, el cine comparte con la narración su carácter mediado, no por la voz del narrador sino por la cámara. Ambos elementos se interponen entre el receptor y la ficción, mientras que, en el teatro, ese elemento intermedio está ausente, el acceso del receptor al mundo de ficción es directo. Se trata, pues, de una representación más objetiva, en tanto que no interviene la subjetividad del elemento mediador.

En lo concerniente a la forma en el análisis automático de textos teatrales, debemos tener muy en cuenta que la \textit{inmediatez} dramática afecta de igual manera al texto escrito que a la representación escénica, a pesar de que, por razones obvias, esta sea menos apreciable en el primero. Sin embargo, esta inmediatez modal encuentra un reflejo estructural externo evidente en el texto, en la superposición de dos subtextos diferenciados, uno descriptivo y otro reproductivo:

\blockquote{Llama sobre todo la atención que en un drama «escrito» discurran dos textos uno junto al otro: por un lado, el texto secundario, esto es, la información sobre dónde acontece la historia representada en cuestión, en qué tiempo, etc., quién está hablando y, quizás también, qué está haciendo, etc.; por otro lado, el texto principal\footnote{En el original se emplea \textit{Haupttext} para el texto principal y \textit{Nebentext} para el secundario.} mismo. Este último consta exclusivamente de frases, que son \textit{pronunciadas} «realmente» por los personajes representados. Mediante la especificación del personaje\index{personaje} correspondiente, las frases del texto principal adquieren una suerte de «comillas». Tanto estas frases mismas como los personajes que se indica que están hablando en cada momento, así como las circunstancias mismas, devienen en lo \textit{representado} a través de los elementos del texto secundario, pertenecen al «nivel concreto» de este. Pero las frases pertenecientes a este nivel son justamente frases y, por lo tanto, constituyen un nuevo nivel, a saber: el de las circunstancias y los destinos de aquellos de los que trata el discurso en las frases pronunciadas. \parencite[p. 220; traducción propia; énfasis en el el original]{ingarden1972}} 

Para \citeauthor{garcia2020}, los términos abstractos \textit{texto principal} y \textit{secundario} encuentran su forma concreta en castellano en \textit{diálogo} y \textit{acotación}. Convenimos con que puede hacerse esta equivalencia \textit{grosso modo}, aunque, en nuestro caso, trataremos de hilar más fino. La manera en que necesitamos estructurar el texto, el personaje y las instrucciones escénicas, incluso siendo ambas indicaciones del dramaturgo, corresponden a dos entidades diferenciadas: la primera es metainformación para la clasificación de los diálogos en el texto completo, ya que es un atributo de estos, mientras que la segunda no. La acotación no afecta siempre a un solo nivel: se refiere a un parlamento\index{parlamento}, además de la evidente didascalia de personaje, si aquel, por ejemplo, se introduce con la acotación «aparte» o solo a una porción de este si llega una vez iniciado el parlamento, por lo que afecta únicamente esa parte del parlamento o el modo de escenificarlo. La acotación\index{acotación} puede afectar a toda la escena (p. ej., «Éntranse todos»), al acto\index{acto} («Tocan chirimías y corrése la cortina. Aparecen...») o a la obra completa («\textsc{Fin}»), o incluso una parte no dialogada («Desenvaina»).

Esto, como vimos en el capítulo anterior, obliga a tomar una decisión ontológica. Por ejemplo, podríamos considerar los parlamentos no según quién los pronuncia sino si indican apartes, líneas cantadas o leídas. Tal clasificación implicaría una organización  diferente a la hecha —\nolinebreak por personaje\nolinebreak— y determinaría los tipos de análisis que podrían llevarse a cabo con tales datos. No obstante, seguimos topándonos con que las acotaciones constituyen una clase abierta, lo que obligaría a considerar apenas un subconjunto mínimo de ellas, las que indiquen una cualidad del parlamento, y tratar al resto como una entidad diferente única. Si bien podría enfocarse así el texto desde otro punto de vista, nos devolvería a la situación original en lo que toca a los segmentos afectados por las acotaciones. Por este motivo, resultaría más productivo crear una estructura de datos que defina una entidad textual cerrada que adopte valores de dos categorías mutuamente excluyentes, diálogo y acotación, cuya pertenencia la determina que el texto forme parte o no de un parlamento.

Aquí se da una cuestión que nos atañe, se trata de dilucidar quién pronuncia cada parte del texto. En cuanto a los diálogos\index{parlamento}, es evidente que corresponden a cada personaje. ¿Pero qué pasa con las acotaciones? Si, como dijimos, no resulta conveniente emplearlas para clasificar otras entidades en función suya, deberíamos ordenarlas como partes textuales en una categoría alternativa a los diálogos. Después de todo, en el siglo \textsc{xvii} forman ya una estructura \parencite[562]{monzo2020}. El autor ha retrocedido del todo, no es factible, como en la narratología, ubicar la presunta voz autorial en el mundo ficcional de la narración. Los apartes de la obra, sin entrar en dónde en concreto, no se pronuncian en el universo diegético. No se trata de una descripción de un suceso del que tiene conocimiento el narrador, sino de una prescripción de cómo ha de desarrollarse un evento aún por suceder en un mundo que es ajeno a la voz. Al contrario que en el drama, el universo del narrador épico es ficcional, se halla en  el mundo diegético o en uno de orden superior que lo contiene, pero no por ello menos ficcional. En la representación, por el contrario, la voz autorial no existe y discurre paralela al texto dramático sin intersección alguna.

Esto sigue siendo cierto tanto si oímos una voz narradora extradiegética, como en el ejemplo de \textit{Bande à part} como si se recita un prólogo como en las tragedias clásicas; en el primer caso, la voz corresponderá a un nuevo personaje sin cuerpo, \textsc{Voz} tal vez, cuya mecánica en nada se diferencia de la del resto de actores\index{actor}, aunque no compartan espacio ficcional. En el segundo caso, estamos no solo fuera de la diégesis, sino del universo ficcional. La declamación acontece en el universo no-ficcional o en uno ficcional diferente al de la historia.

\citeauthor{garcia2020}~\parencite*[49]{garcia2020} sostiene la tesis de que las acotaciones no son de «nadie», como contrapunto a \citeauthor{ubersfeld1996} \parencite*[18]{ubersfeld1996}, quien atribuye este subtexto al dramaturgo. Arguye la incapacidad funcional, si bien materialmente posible, de enunciar el \textit{yo} en la acotación, que distingue del paratexto alegando que aquella\index{acotación}, al contrario que el paratexto, sí tiene un autor. Encontramos evidencias tangibles de esto en los propios textos: \citeauthor{monzo2021}~\parencite*[93-94]{monzo2021} habla, por ejemplo, de un caso de conciencia autorial en las acotaciones, en las que un escritor quiere hacerse oír, pero va reduciendo con cada nueva obra el ímpetu inicial por prescribir su criterio escénico. Si no es tan tajante como para hablar de \textit{voz autorial}, sí reconoce la existencia de una \textit{voz dramatúrgica} en las acotaciones.

 Consideremos ahora las indicaciones del director escénico: estas son en el drama de una naturaleza semejante a las indicaciones del comediógrafo, en tanto que externas a la representación, aunque determinantes de lo que en ella sucede. La diferencia estriba en la fijación escrita de las acotaciones y la naturaleza efímera de las órdenes del director. A pesar de ello, ambas funcionan en el drama de la misma manera. El director de escena usará con toda probabilidad la segunda persona del imperativo para dirigir a los actores\index{actor}, o la tercera persona en subjuntivo cuando sea preciso que se traslade una orden. En cualquier caso, no interviene el \textit{yo}, pues el director no está en escena, de la misma manera que tampoco lo está la voz de las acotaciones en el universo donde se desarrolla el diálogo. 

Suavizaremos aquí la tesis. Las acotaciones no las pronuncia nadie, sí, pero nadie en el subconjunto del hecho teatral que es el universo representado, porque esa voz no existe allí. No obstante, resulta más difícil aceptar un \textit{nadie} absoluto y sin matices, sea solo por no negarle la existencia al dramaturgo y al sufrido director teatral. Nos inclinaremos preventivamente, por el momento, por Ubersfeld y Monzó.

Señala \citeauthor{garcia2020}~\parencite*[51-56]{garcia2020}, asimismo, las diferencias semióticas elementales entre la obra escrita y la representada. Mientras que la primera se desarrolla en dos etapas diferenciadas, una de producción, en la que el \textit{autor} produce la \textit{obra}, y otra de consumo, en la que la \textit{obra} es leída por el \textit{lector}, la obra representada es un proceso único en el que el \textit{actor}\index{actor} representa para el \textit{público}\index{público}, de manera que los procesos de creación y recepción constituyen una única experiencia intersubjetiva compartida por público y actor. Esto no supone únicamente la ausencia en el proceso del \textit{autor}, sino de la obra literaria. Al contrario que el libro, que admite infinitas lecturas de un texto inmutable, la representación es única e irrepetible. Eso la emparenta con la música en vivo o el espectáculo circense, mientras que coloca al texto dramático junto al cine o la grabación musical, que pueden copiarse sin límite y verse, oírse o leerse tantas veces como se desee.

Sobre esto precisamente ya había reflexionado Walter Benjamin en su conocido ensayo sobre la reproductibilidad del arte en nuestro tiempo. Este, además, concreta la obra en la máquina\footnote{\textit{Apparat} en el original.}, lo que agrega al término matices semánticos relacionados con la complejidad de la mediación. Este mecanismo es el que, de forma metafórica, presentan las escenas de las obras narrativas, donde el autor fija su \textit{cámara} discursiva en aquellos detalles que  quiere resaltar y nos oculta otros que prefiere escamotearnos, o la música grabada, que es el resultado de una mezcla particular de la pista de cada instrumento.
\blockquote{Así, la ejecución del actor está sometida a una serie de exámenes ópticos. Esta es la primera consecuencia de que la del actor se presente a través del equipo técnico. La segunda consecuencia radica en que  el actor cinematográfico, al no presentar él mismo su ejecución al público\index{público}, pierde la posibilidad reservada al intérprete escénico de adaptarla  al auditorio durante la representación. Esto se traduce en un carácter de examinador, que no se altera en lo más mínimo por el contacto personal con el actor. \textit{El público solo se identifica con el actor en tanto que lo hace con la máquina. Este asume así su carácter: examina}. \parencite[pp. 37-38; traducción propia;énfasis en el original]{benjamin2013}}

Como consecuencia, es imposible revertir el proceso de la producción teatral. La representación no admite reducción a forma escrita y tampoco acepta ser narrada, ni verbalmente ni de forma visual con una cámara. La esencia del teatro es la actuación: mientras que un encuentro deportivo puede celebrarse en ausencia de público\index{público} sin detrimento de su naturaleza, el teatro lo es solo en la medida que existe la comunicación intersubjetiva entre el actor\index{actor} y el público. «El teatro necesita al público para ser; las demás actuaciones lo requieren para \textit{ser actuaciones}» \parencite[p. 59;énfasis en el original]{garcia2020}.

A pesar de todo, nuestro análisis toma el texto escrito como objeto. Debemos, por lo tanto, considerar la manifestación escrita a la luz de lo expuesto no como la obra, sino una descripción de esta. El texto es, pues, la guía inmutable a partir de la cual se construyen las obras teatrales, sin que ninguna de ellas sea exactamente igual a otra. Retomando lo que tratábamos en el capítulo anterior, el texto es un modelo de la representación y, como tal, parcial e incompleto. De esta manera, describiremos diferentes aspectos de la obra teatral según la parte que consideremos.

De acuerdo con los tipos propuestos por la dramatología, encontramos textos lingüísticos y no lingüísticos. Entre los primeros distinguiríamos orales, como una grabación de los diálogos, y escritos, como las acotaciones. Entre los segundos, podríamos encontrar textos icónicos, como un \textit{storyboard}, y simbólicos, como la partitura de la música. Se clasifican también según si son de referencia verbal, como los diálogos, o no verbal, como la escenografía, así como entre transcriptivos o reproductivos, como el diálogo (sea escrito o grabado), y descriptivos, como las acotaciones \parencite[64]{garcia2020}. Para nuestro estudio, nos interesan los textos lingüísticos escritos, de referencia verbal, tanto transcriptivos como descriptivos, como objeto y medio respectivamente. 

El modelo dramatológico distingue tres categorías en los textos dramáticos:
\begin{itemize}
	\item texto escénico\index{texto!escénico}
	\item texto dramático\index{texto!dramático}
	\item texto diegético\index{texto!diegético}
\end{itemize}
El \textit{texto dramático} es un texto lingüístico de referencia verbal y no verbal, reproductivo y descriptivo, cuyo ejemplo por antonomasia es la propia obra dramática escrita; el \textit{texto escénico} representa aspectos no dramáticos que han de tenerse en cuenta en la representación, como, por ejemplo aquellos que describen los libretos de luces o de tramoya; el \textit{texto diegético} o \textit{fábula}\index{fábula|see {texto diegético}} es el argumento de la obra textualizado \parencite[65]{garcia2020}. Debemos notar que las divisiones no son completamente estancas, ya que resulta habitual incluir en el texto dramático elementos del texto diegético o del texto escénico. 

\section{Estructura del drama}
El libreto de una pieza teatral moderna suele presentar una distribución en actos\index{acto}, subdivididos en escenas\index{escena}. Es responsabilidad del director escénico la organización interna de estas divisiones externas. Esta forma de segmentación es —\nolinebreak en cierto modo\nolinebreak— moderna, pues su redescubrimiento tras el lapso medieval se lo debemos al humanismo renacentista y su interés por Séneca y otros comediógrafos clásicos. La disposición externa responde a una necesidad práctica: el director de escena ha de conocer los personajes que hay sobre el tablado en cada momento durante la representación \parencite[170]{kayser1992}. ¿Qué pasa entonces, por ejemplo, con buena parte del teatro áureo en la que el escritor no señala las escenas? En ese caso, aparte de la organización interna, también las divisiones externas menores reacaerían sobre el director de escena.

Empecemos, pues, por definir la estructura del drama como obra literaria. En un primer nivel hacemos la clasificación tradicional entre \textit{texto}\index{texto} y \textit{paratexto}\index{paratexto}. En este último encontraríamos elementos como el prólogo y los elogios o, en las ediciones modernas, la introducción y notas. El \textit{texto}, por su parte, lo constituirían los dos subtextos de la obra teatral, esto es, el \textit{diálogo} y las \textit{acotaciones}. El diálogo es el componente verbal y las acotaciones el no verbal. Entre estas últimas, distinguimos las acotaciones \textit{espaciales}, que atañen al decorado; \textit{temporales} para hacer indicaciones sobre el ritmo; \textit{sonoras} para señalar las intervenciones musicales o ruidos dentro y \textit{personales}. Estas últimas, a su vez, se dividen entre nominativas, paraverbales, corporales, psicológicas y operativas \parencite[75-79]{garcia2020}.

Monzó \parencite*[75-76]{monzo2019}, por su parte, propone acotaciones de \textit{loci scænici}, que comprenden las salidas y entradas; \textit{de caracterización del personaje}, con indicaciones de vestuario, dramáticas o de utilería; \textit{escenotécnicas}, sobre tramoya y mobiliario; \textit{de técnica actoral}, que guiarían la proxémica, kinésica, gestos y voz y, por último, acotaciones \textit{de efectos especiales}, incluyendo efectos visuales y sonoros, luminotecnia, música y otros.

 En nuestro trabajo, examinaremos con detenimiento los diálogos, pero nos valdremos de las acotaciones para obtener información dramática no verbal. Adelantamos que las sonoras nos ayudarán a determinar el metro, mientras que las personales nominativas y operativas esclarecerán la configuración escénica\footnote{En febrero de 2023, Clara Monzó propuso en Graz una taxonomía más exhaustiva para las acotaciones del drama áureo, sobre lo que propone en su tesis doctoral. Convendría tener en cuenta esta u otra taxonomía igualmente descriptiva tanto para una hipotética extensión de este trabajo como para definir las etiquetas necesarias para trasladarlo a ediciones \ac{xmltei}.}.

El drama se divide en segmentos\index{segmentación} mayores que constituyen los \textit{actos}\index{acto}, que suele corresponder a una unidad espacio-temporal con los límites definidos tanto en el texto como en su representación. En los actos\index{acto} distinguimos instancias de «una acción escénica ininterrumpida que tiene lugar en un espacio y tiempo determinados»~\parencite[91]{ruano2007}, denominados \textit{cuadros}\index{cuadro}. Estos se identifican atendiendo a diferentes marcadores. Algunos de ellos son escénicos, como la situación de que no queden personajes sobre el escenario, que cambie la ubicación o el tiempo de la acción\index{acción dramática}, que se revele un nuevo decorado, o de otro tipo, como que ocurra un cambio estrófico \parencite[291-292]{ruano1994}. El primer tipo de cuadro, que implica un cambio en la configuración de los personajes\index{personaje}, se ha denominado tradicionalmente \textit{escena}\index{escena}.

Los actos\index{acto} están asociados a la ficción dramática, pero los consideraremos como componentes estructurales externos a efectos prácticos, aprovechando los elementos paratextuales que señalan sus límites. Esta demarcación explícita no se da necesariamente en todos los segmentos —\nolinebreak véanse, por ejemplo, los cuadros y escenas sin marcar\nolinebreak—\nolinebreak, si bien es la norma para los de orden superior, por lo que haremos uso de ella y segmentaremos atendiendo al criterio del editor; en ausencia de marcas, la división está sujeta a elementos no siempre apreciables en la estructura propiamente textual, lo que requiere una lectura atenta e, incluso así, las divisiones no siempre están exentas de controversia. Nos centraremos, por lo tanto, en aquellos elementos visibles de la superficie del texto. Recordemos que este trabajo no aspira a resolver el problema la segmentación implícita, sino aportar datos y herramientas que contribuyan, entre otras cosas, a echar luz sobre él, a contribuir a los análisis con metainformación dramática y métrica de los versos\index{verso}. Tengamos en cuenta esta limitación del alcance no solo como prevención, sino como sugerencia para potenciales usos de sus resultados.

\subsection{Acto}

El acto\index{acto} se define como una división externa, mas solamente hasta cierto punto, dado que, en su encarnación clásica es consustancial a la historia. En efecto, este segmento como división dramática engloba acciones que suceden en una unidad espacial\index{acción dramática} \parencite[170]{kayser1992}. Este concepto de \textit{unidad espacial} se ha querido respaldar a menudo apelando a Aristóteles u Horacio —\nolinebreak al menos desde Trissino\nolinebreak—\nolinebreak, a pesar de que la evidencia textual de ello es cuestionable. Lo atribuye \citeauthor{corneille1862}~\parencite*[117]{corneille1862} a una extrapolación espuria de la unidad tiempo a la distancia que se cubre en ella. En efecto, aconseja Lope que se componga el acto «procurando, si puede, en cada uno / no interrumpir el término de un día»  \parencite[213-114]{vega2006}. Del paso de la unidad temporal a la espacial encontramos rastros al final de la primera parte del \textit{Quijote}, en medio de las invectivas del cura y el canónigo contra las tendencias literarias de moda, donde el primero alude a las comedias y, entre otros defectos, les achaca la siguiente falta:
\blockquote{¿Qué diré, pues, de la observancia que guardan en los tiempos en que pueden o podían suceder las acciones que representan, sino que he visto comedias que la primera jornada comenzó en Europa, la segunda en Asia, la tercera se acabó en África, y así se hubiera hecho en todas las cuatro partes del mundo? \parencite[624]{cervantes2016c}}

Resulta irónico que el intransigente cura cervantino se refiera a los actos\index{acto} como \textit{jornadas}\footnote{Entenderemos \textit{jornada} en lo sucesivo como un caso particular del acto. En cualquier caso, nos referimos a conceptos y no a sus denominaciones particulares, por lo que, por ejemplo, los veintiún \textit{autos} de la \textit{Celestina} —\nolinebreak una obra de las que más bebió la literatura aurisecular\nolinebreak— los entenderíamos como \textit{escenas} a pesar de su nombre.}, ya que esta innovación se define, precisamente, por la ruptura con las unidades clásicas. Luis Alfonso de Carvallo ilustra esta característica de la siguiente manera en uno de sus diálogos:

\blockquote{Jornada es nombre italiano; quiere decir cosa de un día, porque \textit{jiorno} significa el día. Y tómase por la distinción y mudanza que se hace en la comedia de cosas sucedidas en diferentes tiempos y días, como si queriendo representar la vida de un santo, hiciésemos de la niñez una jornada, de la edad perfecta otra, y otra de la vejez. A estas jornadas llaman los latinos \textit{actos} (Horatius in \textit{Poetica}) y tiene cinco cada comedia. \parencite[p. 261; énfasis en el original]{carvallo1997}}

Fausta Antonucci \parencite*[112]{antonucci2017} trae a colación a Juan de la Cueva, quien ofrece una razón de peso para este cambio: \blockquote{Huimos la
observancia que forzaba / a tratar tantas cosas diferentes / en término de un día que se daba} \parencite[en][144]{sanchez1972}. En efecto, la jornada dota al drama español de una flexibilidad inusitada porque libera al poeta de la necesidad de embutir la acción en el corsé temporal prescriptivo y, por lo tanto, limitar la acción o poner en riesgo la verosimilitud\index{acción dramática}.

Cuando el cura del \textit{Quijote} vuelve a la carga, nótese que en esta nueva ocasión lo hace arguyendo, nada menos, aquello que usa Carvallo como elemento definitorio de la jornada: \blockquote{Porque, ¿qué mayor disparate puede ser en el sujeto que tratamos que salir un niño en mantillas en la primera cena [\textit{sic}] del primer acto\index{acto}, y en la segunda salir ya hecho hombre barbado} \parencite[624]{cervantes2016c}. No embargante la prescriptiva que esgrime nuestro «hombre docto graduado en Sigüenza»\footnote{La ironía de la frase que emplea Cervantes para caracterizar a su cura ha hecho correr ríos de tinta. La materia la tocamos también nosotros en otro sitio \parencite{sanz2022c,sanz2022a} tangencialmente, aunque no motivados por el clérigo cervantino, sino por la propia ciudad del Doncel y cómo había llegado a adquirir tal «prestigio» la universidad menor allí emplazada.}, nos quedaremos con Carvallo.

Sea como fuere, la comedia nueva acaba con la preceptiva temporal. Lo hace deliberadamente, consciente de los preceptos que contraviene, pero también de la imperiosa necesidad de hacerlo. Los espectadores\index{público} del siglo \textsc{xvii} que acuden a ver comedias son exigentes y demandan una fábula que no se condensa de sol a sol. Lope, buen conocedor de su público, no solo lo asume, sino que lo lleva a gala.

\blockquote{No hay que advertir que pase en el período\\de un sol, aunque es consejo de Aristóteles,\\porque ya le perdimos el respeto\\cuando mezclamos la sentencia trágica\\a la humildad de la bajeza cómica.\\Pase en el menos tiempo que ser pueda\\si no es cuando el poeta escriba historia\\en que hayan de pasar algunos años,\\que estos podrá poner en las distancias\\de los dos actos, o si fuere fuerza\\hacer algún camino una figura,\\cosa que tanto ofende a quien lo entiende,\\pero no vaya a verlos quien se ofende.\\¡Oh, cuántos de este tiempo se hacen cruces\\de ver que han pasado años en cosa\\que un día artificial tuvo de término,\\que aun no quisieron darle el matemático!\\Porque, considerando que la cólera\\de un español sentado no se templa\\si no le representan en dos horas\\hasta el jüicio desde el Génesis\\yo hallo que si allí se ha de dar gusto\\con lo que se consigue es lo más justo. \parencite[189-210]{vega2006}}

Volviendo a Carvallo, este menciona que la comedia latina tiene cinco actos\index{acto}, como también son cinco en los que se dividen el drama alemán, francés e inglés tradicionalmente. Sin embargo, las comedias clásicas del teatro ibérico, tanto español como portugués, presentan una particularidad: están repartidas en tres actos. Como suele ocurrir con las cosas de éxito, a este formato no tardaron en salirle numerosos padres. Juan de la Cueva reclama para sí la invención de la siguiente manera:
\blockquote{A mí me culpan de que fui el primero\\que reyes y deidades di al tablado,\\de la comedia traspasando el fuero:\\que el un acto de cinco le he quitado,\\que reducí los actos en jornadas,\\cual vemos que es en nuestro tiempo usado. \parencite[cit. en][143]{sanchez1972}}

Sin embargo, si prestamos oídos a Cervantes, tal vez deberíamos atribuir el mérito al insigne alcalaíno. 
 
\blockquote{Y esto es verdad que no se me puede contradecir, y aquí entra el salir yo de los límites de mi llaneza: que se vieron en los teatros de Madrid \textit{Los tratos de Argel}, \textit{La destrucción de Numancia} y \textit{La batalla naval}, donde me atreví a reducir las comedias de tres jornadas de cinco que tenían. \parencite[94-95]{cervantes2016a}}.

No dudamos de que ambos tuvieran la osadía de hacerlo, aunque, siendo tanto el uno como el otro aficionados al teatro, podríamos cuestionar la buena fe de sus afirmaciones —\nolinebreak no lo haremos\nolinebreak—\nolinebreak. Tanto Lope de Vega~\parencite*[215-217]{vega2006} como Caramuel~\parencite*{caramuel2011} atribuyen la feliz invención a Critobal de Virués. En cualquier caso, los tres irían a rebufo de la \textit{Comedia florisea} de Avendaño, que presentaba esta división apenas había llegado al mundo el capitán. Y a todos ellos se les habría adelantado Antonio Díez al disponer así el \textit{Auto de Clarindo}~\parencite[172]{kayser1992}. Tal cambio externo tendría como consecuencia que la \textit{fábula} estructurada en \textit{presentación}, \textit{intensificación}, \textit{clímax con peripecia}, \textit{declinación} y \textit{desenlace}\footnote{En el alemán original \textit{Einleitung}, \textit{Steigerung}, \textit{Höhepunkt mit Peripetie}, \textit{Fallen der Handlung} y \textit{Lösung}.}~\parencite[102-122]{freytag1894} habría de repartirse en \textit{presentación}, \textit{nudo} y \textit{desenlace}. Nos inclinamos empero a pensar que la innovación de Freytag consiste en hilar fino el nudo al describir la intensificación de la acción hasta el punto de inflexión climático, a partir del cual aquella comienza a declinar\index{acción dramática}.

A despecho de nuestros pioneros ibéricos, la estructura en tres partes \textit{prótasis}, \textit{epítasis} y \textit{catástrofe} no es en realidad innovación española —\nolinebreak o al menos no lo es como un ideal\nolinebreak[4]—\nolinebreak[4], pues ya la habían descrito —\nolinebreak si tal vez más respecto a la fábula que a la forma exter\nolinebreak{na—} Elio Donato en sus comentarios a Terencio, así como Evanthius en el siglo \textsc{iv} en un tratadillo que se incorporó a los textos del primero. Además del prólogo, ambos gramáticos latinos distinguían tres partes en la actuación que le seguía.

\blockquote{
	La comedia se divide en cuatro partes: \textit{prólogo}, \textit{prótasis}, \textit{epítasis} y \textit{catástrofe}. El prólogo es el primer discurso, que los griegos llamaban πρόλογος, esto es, las palabras que preceden a la verdadera recitación de la historia. Los hay de cuatro clases: συστατιχὸς, comendaticio, que preconiza la obra o al poeta; ἀναφοριχὸς, relativo, por el que se denosta al rival o se elogia al público\index{público}; ὑπoθετιχὸς, argumentativo, exponiendo el argumento de la historia; μιχτὸς, mixto, conteniendo todas las anteriores. Entre prólogo y prologio hay una diferencia, de acuerdo con algunos: es prólogo cuando el poeta se exculpa o se recomienda la historia al público. Sin embargo, es prologio cuando se dice algo sobre el argumento. La prótasis es el primer acto\index{acto} de la historia y el inicio del drama, en el que se desarrolla parte del argumento, pero otra parte se reserva para mantener la expectación de los espectadores. La epítasis es el desarrollo y progreso de las dificultades, por así decirlo, el nudo del enredo. La catástrofe es el retorno de las cosas hasta un final divertido, una vez que los personajes toman conciencia de los eventos. \parencite[p. xviii; traducción propia;énfasis en el original]{evanthius300}
}

Podríamos pensar, sin embargo, que más que una recuperación del formato en tres partes, el teatro español lo reinventa al plasmar la estructura implícita en la forma externa propia del género. No solo por cuanto Cervantes no apela a la autoridad de los clásicos, como convenía hacerse en estos casos para dar más empaque a una \textit{innovación}, sino porque las fuentes clásicas se refieren estrictamente a una división interna. Se afirma en la introducción atribuida a Donato de la terenciana \textit{Adelfos} que \blockquote{esta obra, como otras de su género, tiene necesariamente cinco actos} \parencite[pp. 3-4; traducción propia]{donatus300b}. A pesar de esto, vuelve a reivindicar la segmentación terciaria: \blockquote{los doctores de antaño los separaban [...], el prólogo es más calmado, se ocupa más de justificarse a sí mismo que de atacar al adversario; la \textit{prótasis} es turbulenta; la \textit{epítasis} ruidosa y la \textit{catástrofe} suave} \parencite[p. 4; traducción propia]{donatus300b}, lo que concuerda con la máxima horaciana:\blockquote{No tenga menos ni sobrepase los cinco actos / la obra, que quiere ser demandada, vista y repuesta} (\ars{vv. 189-190; traducción propia}).

Esta falta de correspondencia exacta entre la segmentación externa y la interna no se le pasó por alto a Caramuel, quien delimitaba claramente ambas divisiones. Nótese que, sin embargo, habla de catástasis en lugar de catástrofe y no de una división cuaternaria en la que la catástasis es el segmento que precede a la catástrofe.

\blockquote{Sé que las comedias de los antiguos se repartían en muchos actos\index{acto} y los actos en escenas\index{escena}, mas toda esa cantidad de escenas parece superflua, dado que toda comedia tiene —\nolinebreak y eso mismo pensaban los escritores antiguos\nolinebreak— tres partes: \textit{prótasis}\index{prótasis}, \textit{epítasis}\index{epítasis} y \textit{catástasis}\index{catástasis}, pudiendo cada una corresponder a un acto. \parencite[289-290; énfasis en el original]{caramuel2011}}

En lo que respecta a la obra fijada por escrito, contamos con la inestimable labor de la ecdótica para discernir las divisiones más complejas de la obra. Sin embargo, resulta muy conveniente para nuestro propósito que todas las ediciones de textos dramáticos, críticas o no, se hayan ceñido tradicionalmente a unas convenciones de composición tipográfica precisas, aunque se han ido refinando a lo largo del tiempo. De esta manera, en mucha mayor medida que otros géneros, en el drama pueden identificarse incluso visualmente la estructura de la obra y, al nivel más alto, estarían los actos\index{acto} o \textit{jornadas}\index{jornada|see{acto}}.

Con esto, tendríamos las divisiones mayores de la comedia. Sin embargo, es posible —\nolinebreak al menos en teoría\nolinebreak— dividir estos segmentos globales en unidades menores atendiendo a diferentes criterios, como veremos a continuación.

\subsection{Cuadros}
Como decimos, los cuadros\index{cuadro} se determinan de varias maneras. En su forma clásica, se atienen a criterios escénicos; esto es, son lo que se ha venido denominando como \textit{escenas}\index{escena}. Debemos cuidarnos, sin embargo, de tomar estas divisiones según «el modelo pseudoclásico de las escenas a la francesa, cuya inadaptación manifiesta al objeto teatral áureo no impide que continúe, aún en nuestros días, su maléfica influencia»~\parencite[46]{vitse1998}. Por consiguiente, entenderemos la escena de la comedia nueva no como un recurso poético, sino como un constructo crítico —\nolinebreak una manifestación más de la sempiterna propensión humana por encasillar, que nos lleva no pocas veces a la pareidolia\nolinebreak— o escénico, para satisfacer las demandas organizativas de la función.

No en vano, la escena en su sentido clásico es una subdivisión externa del acto\index{acto} que responde a una necesidad práctica real. La segmentación facilita el control de los personajes que hay en cada momento sobre las tablas durante la representación. Esto cobra una especial importancia en montajes de compañías pequeñas o cuando pocos actores\index{actor} han de asumir varios papeles en la misma pieza \parencite[170]{kayser1992}. Un cambio de decorado —o, en los teatros de comedias áureos, mudar una \textit{apariencia}— o incluso correr un telón requerirían asimismo cierto control. Sería, no obstante, más discutible la influencia del criterio si el acto se ciñe al precepto de la unidad de tiempo.

Hay que advertir asimismo de que ninguno de estos criterios reclama para sí la predominancia absoluta en la determinación de la estructura. Los criterios escénicos no carecen de imprecisiones, mientras que los sonoros suelen resultar demasiado exiguos. Por lo tanto, ninguno de ellos resulta autosuficiente porque todos requieren el apoyo de elementos estructurales internos, apelando a la línea temporal, el espacio o la acción. Ni que decir tiene que estos son, con harta frecuencia, hipotéticos, en el mejor de los casos \parencite[127]{oleza2010}.

De cualquier manera, la segmentación como respuesta a una necesidad del  montaje entra en conflicto con otra exigencia también de orden práctico. Dice Donato que \blockquote{en un intento por retener a un espectador mal dispuesto, el dramaturgo latino minimiza las divisiones, temiendo seguramente que el espectador, aburrido al final de un acto\index{acto} o incitado a irse, rechace el resto de la comedia y se levante} \parencite[pp. 3-4; traducción propia]{donatus300b}. Los contemporáneos del crítico latino encontrarían una equivalencia directa en el público\index{público} barroco, al menos en lo concerniente al nerviosismo del respetable ante el escenario vacío. Contravenir los gustos de la audiencia no es cosa que deba tomarse a la ligera: incluso el \textit{Arte nuevo} previene sobre la posibilidad y recomienda evitarla en la medida de lo posible.

\blockquote{Quede muy pocas veces el teatro\\sin persona que hable, porque el vulgo\\en aquellas distancias se inquïeta\\y gran rato la fábula se alarga;\\que, fuera de ser esto un grande vicio,\\aumenta mayor gracia y artificio. \parencite[240-245]{vega2006}}

Este temor al público lleva a menudo a eslabonar segmentos sin solución de continuidad, lo que produce una liminaridad difusa y, si a las palabras de Donato hemos de atenernos, también confusa. Este escribe lo siguiente al respecto, aunque hay que prevenir de que el comentarista no se refiere a las escenas\index{escena}, sino a los actos\index{acto}:

\blockquote{Es difícil discernir con claridad las divisiones de los actos\index{acto} en las obras latinas, pues están ocultas [...]. Pero merece la pena aprender —\nolinebreak si bien con dificultad\nolinebreak— a discernir  cómo y con qué criterio es posible entenderlas y distinguirlas. Empezaremos diciendo que un personaje que ha salido a escena cinco veces no puede salir otra vez más. Con frecuencia nos engañamos pensando que un personaje, por estar en silencio, ha entrado en escena, cuando espera sin hablar en el proscenio el momento de dar su réplica. Resulta, por lo tanto, necesario observar con cuidado dónde y cuándo se vacía el escenario de todos los personajes, de manera que se escuche al coro o el flautista. Cuando vemos eso, reconoceremos ahí el final de un acto. \parencite[p. 5; traducción propia]{donatus300c}}

Esta concepción deliberadamente imprecisa de las divisiones hace que, en la práctica, resulte muchas veces dificultoso discernirlas en el montaje\footnote{Marcus~\parencite*[327-328]{marcus1973} llama la atención sobre un particular que introduce cierto grado de subjetividad interpretativa en el asunto. Saca a colación la \textit{opera prima} de Schiller~\parencite*{schiller_raeuber} \textit{Die Räuber}. En un alarde pasional de \textit{Sturm und Dränger}, el poeta mata a una serie de personajes uno detrás de otro en las tres últimas escenas de la obra: Spiegelberg, Franz, Schweizer, el conde de Moor y Amalia. Marcus  considera un cambio de configuración dentro de la escena, dado que el personaje está ausente —\nolinebreak aún de cuerpo presente\nolinebreak[4]—\nolinebreak[4]. No da indicaciones Marcus respecto al caso de que, dado que Spiegelberg muere en el mismo lugar en el que sucede la última escena de la obra, bien podría considerar el directo de escena volver a poner al actor allí tendido\index{actor}. De acuerdo con su interpretación de las configuraciones,  cabe suponer que no habría que considerarlo como un personaje sino como parte del decorado. ¿Y los personajes que, como los bandidos en la obra, duermen roncando sonoramente, y \textit{entran en escena} al ser despertados por la detonación de un arma? ¿Y si descansaran en silencio?}. También ocurre esto en el texto escrito, donde suele dejarse la tarea al buen entendimiento del lector lo que se dispone sobre el tablado a discreción del director escénico.

Lo que es cierto para separaciones explícitas como los actos\index{acto}, lo es con más razón para las implícitas. Existen ciertas convenciones que el público\index{público} o el lector han de tener en cuenta para analizar la estructura externa de la obra e interpretarla de manera correcta. Si hablamos de actos en el teatro latino, Donato vuelve en nuestra ayuda, si bien resulta solo de utilidad en una versión preceptiva ideal del segmento. 
\blockquote{Con frecuencia confunde al lector que un personaje pronuncie la última línea de un acto\index{acto} y la primera del siguiente sin percatarse de que ha entrado [de nuevo], lo cual notan los expertos de inmediato por la secuencia de sucesos y el lapso de tiempo. Puede pasar, pues, que un personaje salga y entre cuando creemos equivocadamente que no se ha movido del proscenio. Un personaje puede salir a escena cinco veces, pero no decimos que deba hacerlo, se trata de dejar claro que no puede salir una vez más. En la tragedia se permite y es usual salir menos veces. \parencite[p. 5; traducción propia]{donatus300c}}

En la práctica, el aviso de Donato se toparía con un contratiempo que imposibilitaría su aplicación al teatro áureo. La división implícita de la jornada en escenas da lugar a que los personajes tengan oportunidad de entrar y salir un número indefinido de veces en cada acto\index{acto}. De esta manera, la segmentación del drama latino según Donato estaría más emparentada, al menos en este sentido, con la hipotética escena de la comedia nueva que con la jornada, y es en ese ámbito donde encontraría su aplicación. 

En cuanto al reflejo de las escenas en el texto escrito, no nos acompaña la misma suerte que con los actos\index{acto}. El teatro del siglo \textsc{xvii} suele carecer de marcas específicas para señalar estas divisiones en las ediciones impresas, aunque se encuentran manuscritos en los que se indican de forma explícita, por ejemplo, mediante una línea trazada entre las escenas.

Como indicamos, la obra teatral se divide también según criterios no escénicos; entre estos se cuenta la segmentación de la obra según el metro de los parlamentos~\parencite[20]{vitse2010}\index{parlamento}. Aunque no pretendemos definir aquí un método para encontrar estas divisiones ni considerarlas, pues requeriría evaluar factores internos, que exceden las intenciones de este trabajo, sí aspiramos a facilitar en la medida de lo posible la tarea. En otras palabras, el resultado de este estudio ha de proveer una vía para señalar los cambios estróficos y presentar los resultados de la manera más neutral posibles. Es ya tarea del exégeta interpretarlos y determinar cuáles de esos cambios marcan una división más allá de lo puramente métrico. No entramos en los criterios que se consideren pertinentes a la hora de utilizar esta información. De ahí que no consideremos segmentar los textos en cuadros, sino organizar nuestros resultados de manera que  ayuden a hacerlo. Esto es, no seremos nosotros los que hagamos estudios de segmentación, pero necesitamos conocer el tema y adaptar nuestro trabajo sus necesidades para producir resultados útiles.

Por otra parte, suscita hoy pocas dudas que hay una relación entre la naturaleza semántica de los pasajes el su metro de una buena composición teatral. El \textit{Arte nuevo} ya da buena cuenta de ello cuando aconseja adecuar los versos a la materia de la que se ocupan.

\blockquote{Acomode los versos con prudencia\\a los sujetos de que va tratando.\\Las décimas son buenas para quejas;\\el soneto está bien en los que aguardan;\\las relaciones piden romances,\\aunque en octavas lucen por extremo;\\son los tercetos para cosas graves\\y para las de amor, las redondillas. \parencite[305-312]{vega2006}}

Esto parte ya del proceso compositivo, el poeta con buenos hábitos de trabajo no escribiría de ordinario la fábula en versos a vuela pluma, sino que dispondría primero sus partes en prosa, para versificarlas después: «El sujeto elegido escriba en prosa / y en tres actos\index{acto} de tiempo le
reparta»~\parencite[211-212]{vega2006}, lo que implica interrupciones en el planteamiento de la versificación. Dicho de otra manera, dos segmentos en prosa contiguos tienen cada uno versificación propia, sea coincidente con la otra o no. Aunque, como recuerda Fausta Antonucci~\parencite*[78]{antonucci2010}, la segmentación métrica ya se había abordado con anterioridad (\cite{williamsen1978}; \cite{dixon1985} y \citeyear{dixon1994}; \cite{vitse1996} y \citeyear{vitse1988}), esta cobró una especial relevancia teórica a partir de un artículo del último sobre el \textit{Don Juan} \parencite{vitse1998}.

En este estudio, Vitse arguye que el metro tiene una función estructurante y, para ello, se sitúa «preferentemente en el texto escrito, con especial atención al
texto en el momento de su escritura, y con clara y consciente postergación de la representación». Así se plantea «por qué, en un momento dado cambia un poeta de comedias la forma métrica hasta entonces utilizada por él» \parencite*[49]{vitse1998}, no considerando las formas sino «el fenómeno del cambio en sí»~{(p. 50)}. De sus observaciones, colige Vitse que el cambio estrófico no es solamente un elemento opcional del paso de un cuadro a otro, sino una característica necesaria~{(p. 52)}.

Advierte, empero, de que, siendo el cambio estrófico condición necesaria, no es suficiente para determinar un cambio de cuadro. De ahí que no solo proponga el metro como elemento delimitador de las \textit{macrosecuencias}\index{macrosecuencia}, sino que identifique su función estructurante de las \textit{microsecuencias}\index{microsecuencia} de las que se componen las grandes unidades. Esto es, \textit{secuencias englobadoras} y \textit{secuencias englobadas}. Va incluso un paso más allá y hace la siguiente afirmación sobre \textit{El burlador de Sevilla}:

\blockquote{A nivel de las microsecuencias, pues, y en este momento de nuestro análisis, no le pertenece la primacía al elemento espacial, que vendría confirmado eventualmente por algún que otro corte métrico interno. No, ahora, sobre todo cuando estalla entre ellas un conflicto de poder divisorio, quien manda es la métrica, y es su vasalla la repartición espacial configurada por los desplazamientos, hacia fuera o hacia dentro del escenario, de los personajes\index{personaje}. \parencite[53]{vitse1998}}

Arguye que la comedia nueva se estructura en torno a la acción\index{acción dramática}, a la que se supeditan los aspectos espaciales y temporales; las marcas divisorias entre distintos bloques de acción se señalan mediante el metro, que coincide o no con cambios espacio-temporales \parencite[55]{vitse1998}. En consideración a sus observaciones, Vitse propone el siguiente modelo interpretativo (pp. 58-59):

\begin{enumerate}
	\item Ha de partirse de la premisa de que el dramaturgo emplea la pluralidad métrica tanto para fines estéticos como estructurales.
	\item Con esto en mente, han de jerarquizarse las formas métricas englobadas y englobadoras en el texto.
	\item Una vez sistematizado, se identifican las unidades métricas básicas o \textit{secuencias}.
	\item Con apoyo de los criterios espaciales, temporales, escénicos y escenográficos, se establece la distribución definitiva en macrosecuencias monométricas, macrosecuencias polimétricas sencillas y macrosecuencias polimétricas complejas.
\end{enumerate}


La distinción entre los dos tipos de macrosecuencia polimétrica se halla en la concurrencia o no del resto de criterios, así como de otros indicadores externos para distinguir los límites de cada macrosecuencia fuera de toda duda. Esto aconseja una postura más conciliadora, como la propuesta de Crivellari~\parencite*[82]{crivellari2013} de una interacción hermenéutica entre criterios. En la práctica, parece una aproximación sensata, como reconoce  \parencites[38-46]{casariego2018}.

En \textit{La cena del rey Baltasar} de Calderón\nocite{calderon_baltasar}, encontramos un ejemplo práctico de en qué nos afecta esto y las limitaciones de la automatización de los análisis digitales. \citeauthor{sanchez2013}~\parencite*[39-54]{sanchez2013} segmentan el texto en una macrosecuencia inicial (vv. 1-1224) y otra final (vv. 1225-1574). En la segunda, distinguen tres divisiones estructurales: una de silvas pareadas (vv. 1225-1322) con una copla englobada (vv. 1313-1316), una con un romance en \textit{e-o} (vv. 1329-1380) orlada por sendas sextillas (vv. 1323-1328 y 1381-1386) y una última con un romance en \textit{e-o} (vv. 1387-1574). Incluso conociendo los límites de las secuencias, el crítico debe escrutar la evidencia interna para determinar que la última sextilla pertenece al mismo segmento que el penúltimo romance, pero a uno diferente que el último.

Y, al contrario, el último segmento de la primera macrosecuencia (vv. 1074-1224), salvo unos pocos versos englobados cantados, es un romance en \textit{e-o} y, por lo tanto, métricamente indistinguible del primero de la segunda. Considerar los cambios escénicos puede no resultar suficiente; vayan como muestra los últimos versos (vv. 1190-1224) de la secuencia del sueño (vv. 1074-1224), que ocurren tras despertar Baltasar. Estos comparten ubicación con la macrosecuencia siguiente, diferente del espacio onírico de la anterior. Atendiendo meramente al metro, espacio y vacíos escénicos, esos versos deberían integrarse en el segmento siguiente. Sin embargo, en cuanto a contenido, no cabe duda de que es un epílogo al sueño, por lo que tendría poco sentido hacerlo. Apenas contamos con la salida de la Idolatría como marca divisoria; así, careciendo del juicio sobre la evidencia interna que aporta el criterio humano, sería harto complicado determinar cuándo señala un cambio de secuencia la salida de un personaje a escena.

A esto debemos añadir el lugar que ocupan las partes cantadas en el esquema. Gilabert~\parencite*[84]{gilabert2014} aboga por reconocer la existencia de macrosecuencias musicales, a las que atribuye una función estructural y capacidad de englobar otras partes menores, sean cantadas o declamadas. Aquí tenemos menos opciones porque solo podemos señalar divisiones si hay un cambio métrico, como en los versos ordinarios, pero no en el caso de música instrumental, que requeriría un escrutar las acotaciones con más atención de la que aquí podemos ofrecer. No obstante, el análisis automático y clasificación de las acotaciones podría ser un interesante campo a explorar en el futuro.

En resumen, a pesar de que no se establece una distinción explícita en los textos áureos suficiente para llevar a cabo la segmentación, es posible realizarla atendiendo a algunas características. Entre estas, se cuentan algunos elementos de orden externo, como el vacío escénico y el cambio estrófico, por lo que un texto etiquetado con dicha metainformación podría dividirse de forma automática según una u otra característica o una combinación de ambas, lo que reduciría el proceso manual a cotejar los elementos externos para verificar la idoneidad de las divisiones. El vacío escénico presenta algunas  dificultades porque la automatización debe guiarse por las siempre parcas acotaciones. Esto es, identificar una configuración escénica a partir de una acotación o el escenario vacío llevando la cuenta de la salida de los personajes del escenario\index{escenografía}. Resultaría mucho más difícil considerar formas más sutiles de señalar la nueva secuencia, como la desaparición de un personaje. El cambio estrófico, por el contrario, se observa directamente con independencia de la estructura interna o del detallismo del poeta para describir la escenografía, se trata de un hecho objetivo y explícito. Nuestra labor es, pues, extraer esta información en la medida que lo admita el texto y presentarla de la mejor manera para determinar las lindes de los bloques constituyentes de cada obra dramática.

\subsection{Unidades mínimas}
Del mismo modo que los cuadros componen actos\index{acto}, aquellos también están constituidos por  unidades menores. Estas, claro está, varían en función del criterio observado para segmentar la obra. De esta manera, si nos ceñimos al auditivo, consideraremos el verso\index{verso} como unidad mínima del segmento, mientras que, si nos decidimos por dividir considerando las configuraciones escénicas, la unidad textual mínima será el parlamento\index{parlamento}. Esta última, además, es un elemento común a todos las obras teatrales, en verso o en prosa, y prerrogativa del género. Su modalidad escrita se vale de un elemento compositivo visual para señalar tales divisiones, en lugar de los signos ortográficos con los que se denota el discurso directo en otros géneros. Dado que trabajamos con textos escritos, nos limitaremos a describir estas unidades en el texto fijado, con especial énfasis en las ediciones modernas de las que nos valdremos.

Lo primero que hay que hacer notar es que las piezas teatrales tienen como fin la representación, por lo que los apoyos visuales  tienen una finalidad práctica concreta y no responden solo a consideraciones estéticas. Una composición que defina cada elemento ayuda al actor\index{actor} en la lectura escenificada —\nolinebreak y, para nuestra fortuna, también a nosotros en el proceso de descomponer el texto en sus elementos constituyentes\nolinebreak[4]—\nolinebreak[4]. No llama a asombro, pues, que estas divisiones aparezcan de forma explícita incluso en los autógrafos.

Si tomamos, por ejemplo, el autógrafo de \textit{La dama boba} \parencite{lope1613}, vemos que Lope  configura el manuscrito según una disposición precisa: una columna amplia a la derecha con los parlamentos\index{parlamento} distribuidos en una línea por cada verso\index{verso}, a su izquierda aparece una columna menor con las didascalias de personaje\index{didascalia de personaje} subrayados. Las acotaciones suelen aparecer en una columna más a la izquierda (fol. 2v), ocupando las tres columnas y orlados por líneas de división entre cuadros\index{escena} (fol. 4r). No faltan inconsistencias, como versos fuera de las posiciones asignadas (fol. 5r). Los dramaturgos de la época no suelen diferir sustancialmente de este modelo convencional. Así, apreciamos en un autógrafo de Calderón de \textit{El secreto a voces} \parencite{calderon1642} los mismos elementos, con pequeñas variaciones. Aquí vemos los parlamentos en una columna central principal, y las didascalias de personaje en otra más estrecha a la izquierda, subrayadas en ocasiones, mientras que otras acotaciones\index{acotación} aparecen en el margen derecho.

Las ediciones impresas de la temprana Edad Moderna reúnen ya todos los elementos tipográficos necesarios para analizar la estructura de la obra dramática. Por ejemplo, en \textit{La dama boba} \parencite{vega1617}, encontramos el cuerpo del texto dividido en dos columnas, con la indicación de la jornada\index{acto} de cuerpo entero. En cada una de las columnas se observan tres niveles textuales. Los parlamentos\index{parlamento} aparecen en redonda alineados a la izquierda y a cada línea le corresponde un verso\index{verso}. Las acotaciones van centradas en cursiva y las didascalias de personaje\index{didascalia de personaje} se sangran a la francesa\index{sangrado} en la primera línea del parlamento, con frecuencia como una abreviatura del nombre del personaje. Los versos compartidos\index{verso!compartido} difieren de la convención actual en que no se sangra la continuación en el siguiente renglón, sino que el la didascalia para señalar el personaje continuador del verso, así como la primera línea de su parlamento\index{parlamento}, se añaden a la última línea del parlamento precedente, a continuación del inicio del verso si el espacio da lugar a ello. Las imprentas respetan este mismo modelo en ediciones posteriores, como se aprecia, verbigracia, en \textit{El secreto a voces} \parencite{calderon1795} impreso a finales del siglo \textsc{xviii}, más de cien años posterior a la mencionada de Lope. Se repiten los mismos elementos sin apenas alteraciones.

Las ediciones modernas, como decimos, adoptan estas convenciones con pequeñas variaciones, si acaso con algunas mejoras y añadidos de unos pocos elementos para facilitar la lectura, como la habitual disposición escalonada de las réplicas en los versos compartidos\index{verso!compartido} \parencite[67]{arellano2007}, en lugar de la línea única. A veces, se añaden otras indicaciones ausentes en los manuscritos y ediciones clásicas, como las acotaciones editoriales. Estas van entre corchetes o paréntesis en algunas impresiones modernas para distinguirlas de las propias del texto. También se diferencia entre la acotación externa\index{acotación!externa} de carácter independiente y la interna\index{acotación!interna}\footnote{No confundir con las \textit{acotaciones implícitas} \parencite[38-40]{monzo2019}. Cuando nos referimos a acotaciones en este trabajo, sean internas o externas, lo hacemos a las explícitas.}, que afecta a un parlamento o una parte de este, que suele anteponerse a la primera línea de las que modifica en el mismo nivel de sangrado\index{sangrado} que el resto de líneas del parlamento.

Con estas prevenciones, consideramos entidades clasificatorias concretas, empezando por las metatextuales, esto es, autor\index{autor} para el comediógrafo, género\index{género} —\nolinebreak por ejemplo comedia, auto sacramental o entremés\nolinebreak— y subgénero\index{subgénero} —\nolinebreak de capa y espada, de figurón, etc.\nolinebreak—\nolinebreak. En cada una de estas clases, se incorporan varias piezas teatrales. Por su parte, encontramos divisiones textuales, la mayor de las cuales es la jornada o acto\index{acto} en obras de cierta extensión y género. Por debajo de esta, se hallarían los cuadros que, como hemos dicho, no se indican de manera explícita. Por el contrario, otros elementos de orden inferior son discernibles visualmente en la composición de la página. Se distinguen los parlamentos\index{parlamento}, las líneas\index{línea} que los componen, así como un atributo, el \textit{locutor}\index{locutor}\footnote{Hablamos aquí de \textit{locutor} y no de didascalia de personaje porque ya no nos referimos a la acotación del texto, sino a un atributo de las categorías abstractas que definimos.} que  pronuncia cada uno. Como advertimos, las líneas tienen una correspondencia biunívoca con los versos\index{verso} en ediciones antiguas. Por el contrario, el locutor no es en ellas un atributo único del parlamento ni la línea en su totalidad elemento constitutivo, pues el verso compartido tiene dos locutores y cada una de sus partes pertenece a un parlamento distinto. En las ediciones modernas, por el contrario, esto no se cumple para los versos compartidos, que se dividen en tantas líneas como en parlamentos se distribuyan. 

Las acotaciones también se distinguen en la tipografía y composición\index{acotación}. Dependiendo de su posición relativa respecto al parlamento\index{parlamento}, se dividen en internas\index{acotación!interna} o externas\index{acotación!externa}. La acotación externa estaría al mismo nivel que el parlamento\index{parlamento}, mientras que la interna se subordinaría a él. El parlamento, por su parte, se caracteriza por el locutor\index{personaje} que lo pronuncia y su posición en la serie, y está compuesto de líneas y, ocasionalmente, de acotaciones internas. Las líneas se caracterizan también por el verso\index{verso} al que pertenecen, que, como dijimos, en las ediciones modernas suele corresponderse con la línea, pero no necesariamente. A pesar de ello, estas ediciones facilitan en gran medida la identificación visual, al señalar con niveles adicionales de sangrado\index{sangrado} las réplicas\index{réplica} y contrarréplicas\index{contrarréplica} en los versos que se extienden por varias líneas\index{verso!compartido}. 

\section{Ediciones... digitales}
Al abordar digitalmente los textos áureos, nos enfrentamos a diversos contratiempos y, contra lo que pudiera parecer, la escasez de modelos lingüísticos para analizar la variedad diacrónica es el menor de ellos. No debe llevar esto a sorpresa, porque salvar ese obstáculo no requiere más que tomar una edición moderna con ortografía normalizada. La ortotipografía\index{ortografía}, como sabemos, es una convención relativamente moderna que, en las obras auriseculares, depende en buena medida de los gustos y recursos instrumentales del cajista. Por fortuna, esto no es sino una inconveniencia menor, ya que la finalidad del teatro es —\nolinebreak no lo perdamos nunca de vista\nolinebreak— llevarse a escena. La voz, como es evidente, no transmite la ortografía.

No quiere decir esto que haya unanimidad en la ecdótica, si bien una escuela opta por modernizar las grafías, siempre que esto no afecte a la pronunciación \parencite[36]{arellano2007}, hay críticos que permanecen más fieles a la grafía original. Diferencias aparte, ambas aproximaciones suelen ser respetuosas con el consonantismo del texto y conducirían a una misma transcripción fonética moderna. Superadas las vacilaciones del siglo \textsc{xvi}, el reajuste de las sibilantes está consumado a efectos prácticos en el periodo que nos ocupa \parencite[91-93]{lapesa2008}, por lo que una modernización ortográfica no debería alterar la esencia de la representación escénica.

Donde encontramos un problema con la modernización es en parte de las vocales. Calderón, por ejemplo, escribe en sus autógrafos \textit{cay}, que se vuelve \textit{cae} en ediciones modernas. Esto  da lugar a ciertas complicaciones. Partimos de la premisa de que, como veremos más adelante, las vocales altas /i/ y /u/ hacen diptongo fonológico (ver \ref{sec:diptongo}\index{diptongo}) en interior de palabra, por lo que \textit{cae} sería bisilábica, al contrario que \textit{cay}, que sería monosilábica. En principio, el texto moderno sin advertencia al respecto nos llevaría a dos silabeos diferentes. A nivel fonético podríamos unificar las sílabas porque la versión moderna no forma diptongo\index{diptongo}. Si optamos por esta, encontraríamos un nuevo contratiempo de orden suprasegmental. Tanto el diptongo como la reducción son decrecientes. Por lo tanto, para que pudiera haber una sinalefa con una hipotética vocal en la siguiente palabra —que resultaría en triptongo—, esta debería tener una apertura igual o menor que la semivocal o vocal reducida precedente \parencite[]{navarrotomas1957}. De esta suerte, \textit{cae} podría unirse, además de con palabras comenzando con vocal alta —fonema que se reduciría\textemdash, también con otras que comiencen por \textit{e}, mientras que \textit{cay} solo aceptaría vocales altas. Más complicado lo tiene el editor para satisfacer por igual ortografía y pronunciación en la equivalencia /i/-/e/ en palabras como \textit{fénix} o \textit{fácil}, como ocurre, por ejemplo, en posición de rima en un romance en \textit{e-e} de la \textit{La vida es sueño}.

Los signos de puntuación son un universo en sí mismos. Partimos de manuscritos con una puntuación deficiente las más de las veces \parencite[49-63]{arellano2007}. De aquí, pasamos a una tipografía que varía según las veleidades y equipo de cada cajista, inconsistentes entre imprentas, entre libros de la misma imprenta e incluso, en ocasiones, entre páginas del mismo libro \parencites[142-143]{blecua2001}[99]{carrenno2019}. Gracias a la normalización ortográfica, las ediciones modernas han podido superar estas carencias, así como otras de índole acentual. Aquí es ya el criterio del editor el que fija la puntuación conforme a la norma y las variaciones son deliberadas para reflejar la interpretación del texto. 

Para ilustrar cómo  afecta al ritmo una elección editorial aparentemente mínima, tomaremos prestado un ejemplo de \citeauthor{arellano2007}~\parencite*[64]{arellano2007} en el que trae a colación las formas \textit{Marinatarasca} y \textit{marina tarasca}. Más allá del significado, la acentuación prosódica de \textit{Marinatarasca} sería \texttt{----+-} mientras que la de \textit{marina tarasca} sería \texttt{-+--+-}. No nos cabe duda de que las ediciones críticas modernas aventajan a históricas, en tanto que las primeras atienden a un criterio filológico y no dependen de la intuición del cajista o la disponibilidad de tipos. No podemos sino subrayar la importancia de emplear ediciones bien curadas para obtener los mejores resultados posibles.

Las ediciones modernas tampoco están exentas de inconvenientes. La multiplicidad de formatos digitales, estilos y convenciones hace que extraer la información requiera un proceso distinto entre ediciones, incluso si, a simple vista, las diferencias son mínimas. Hemos hablado ya de la disposición del texto, de la estructura dramática ideal fijada por escrito. Queda, por lo tanto, indicar cómo se plasma esta disposición en las ediciones digitales que queremos examinar. No es esta una cuestión excesivamente compleja en lo que se refiere al propio texto, ya que el trabajo de enjundia ya lo han llevado a cabo otros para llegar hasta allí. Las reglas del juego ya las pactaron dramaturgos e impresores con los lectores en tiempo inmemorial, por lo que poco queda que añadir en ese sentido. El texto en sí, gracias a la encomiable labor de los editores, llega hasta nosotros curado con esmero. El interés para nuestro trabajo radica en la estructura de la información que queremos extraer, así como otra metatextual que produciremos a partir de aquella.

Debemos mencionar en primer lugar algunos aspectos sobre el formato interno de la edición digital del texto dramático. No es esta una cuestión trivial, pues la forma en que se organiza la información determina tanto la manera de encarar la tarea como los usos que se le podrá dar al producto resultante. Para ello, debemos considerar estas ediciones como un conjunto organizado de datos\index{datos}, tal y como los describimos en el capítulo anterior. En este sentido, debemos desembarazarnos de la abstracción con la que vemos el libro electrónico, dejar de entenderlo como una emulación digital del libro de papel y reducirlo a sus partes: el lector de libros electrónicos es un computador equipado con una pantalla en la que representa símbolos —\nolinebreak el texto\nolinebreak— conforme a las instrucciones que encuentra en un archivo informático —\nolinebreak el libro electrónico\nolinebreak—\nolinebreak. Por lo tanto, el computador opera con los datos del libro electrónico representándolos en la pantalla.

Aquí es donde las diferencias entre datos estructurados y no estructurados adquieren protagonismo. Nuestras fuentes serán principalmente ediciones dispuestas como datos no estructurados. Esto es, el libro electrónico, cuya disposición visual reproduce la del libro y que podemos incluso imprimir de manera que el producto final no diferiría cualitativamente de un libro ordinario. La máquina conoce detalles visuales del texto, pero permanece ajena a su estructura interna: sabe que debe colocar la primera letra de la didascalia de personaje en unas determinadas coordenadas espaciales y la segunda a su derecha; la línea inicial del parlamento\index{parlamento} ha de ubicarse en tal posición, y las sucesivas, a la misma en el eje horizontal pero añadiendo un desplazamiento hacia abajo. Estas tres partes son exactamente equivalentes para la máquina, ya que interpreta la posición geométrica como un valor arbitrario. Esto es, la información no está organizada jerárquicamente, lo que dificulta su análisis. Nótese cómo el cerebro humano organiza la información visual para deducir la estructura textual. La tarea consiste, por lo tanto, en enseñar al computador a interpretarla de forma análoga a como lo hacemos cuando leemos la página.

Los textos críticos se presentan también como datos estructurados, o, mejor dicho, semiestructurados. El formato más en boga y que se considera estándar al tiempo de redactar estas líneas es \ac{xmltei}, que ya introdujimos en el capítulo anterior. En este caso, la organización no es estrictamente visual, sino que se encuadra en una jerarquía lógica señalada con etiquetas. De este modo, la etiqueta que señala el texto contendría otras divisiones, por ejemplo, los preliminares y el cuerpo del texto. Estas contendrían otros elementos, como los actos, que, a su vez, estarían formados de escenas, y estas, de elementos menores, como parlamentos\index{parlamento} y acotaciones\index{acotación}. Siguen siendo factibles otras unidades, tanto por encima como por debajo. Así, es posible agrupar los versos\index{verso} en unidades métricas (por ejemplo, englobando los que pertenecen a un romance en una etiqueta al efecto), o indicar información adicional dentro de la jerarquía del parlamento para marcar, verbigracia, la estructura métrica.

De la misma manera, se incluye otra información metatextual, mediante la que se insertaría un aparato crítico al uso, tanto en forma de notas usando la etiqueta apropiada en la jerarquía a la que concierna o de, por ejemplo, unos preliminares, mediante la creación de otra estructura aparte y al mismo nivel que el cuerpo texto. Esta organización es ideal porque, por un lado, existen herramientas para transformarla a un formato visual tradicional para su lectura y, por otro, la jerarquización de la información facilita su procesamiento digital. Lamentablemente, no abundan las ediciones \ac{xmltei} y, entre estas, las que contienen información complementaria metatextual son los menos. Así pues, cobra más importancia nuestra labor de tomar los textos en el primer formato y extraer la información para disponerlos en el segundo u otro equivalente.

\section{Análisis dramático cuantitativo}
\citeauthor{dinu1968}~\parencite*[34]{dinu1968} y  \citeauthor{marcus1973}~\parencite*[316]{marcus1973} coinciden en destacar la labor que requiere llevar a cabo los cálculos envueltos en los análisis dramétricos, pues una pieza de $N$ personajes tiene $2^N$ posibles configuraciones escénicas. La primera jornada de \textit{La vida es sueño}, por ejemplo, con apenas diez personajes, adoptaría $1{.}024$ valores. Llegados al incremento exponencial, el análisis asistido por computador se torna crucial (ver \Cref{chap:ejdram} para un ejemplo de análisis dramétrico manual). Las magnitudes envueltas, si bien pueden obtenerse a mano, requieren recopilar datos y hacer operaciones sencillas para obtener los valores de cada una. En una comedia, tendríamos que ir unidad por unidad\footnote{Macrosecuencia, secuencia, escena, cuadro...} anotando los personajes presentes. A partir de eso, se llevarían a cabo los procedimientos aritméticos necesarios para calcular las medidas relativas para cada par de ellos. Mediante la combinación de estos resultados, obtendríamos los valores absolutos para cada personaje y otras medidas complejas. Si bien una jornada requiere un tiempo relativamente breve y no supondría mayor problema para, por ejemplo, repetir el proceso con las otras dos jornadas y analizar la variación de las configuraciones escénicas a lo largo de la obra si —\nolinebreak como es nuestro objetivo\nolinebreak— automatizamos la extracción de datos y disponemos estos de forma tal que puedan ser usados por el programa informático que realice las operaciones, obtener los valores de toda la obra sería, literalmente, cuestión de segundos.

Si bien tal análisis daría lugar a observaciones interesantes, no variaría sustancialmente de la aproximación tradicional. Después de todo, nos habríamos acercado al texto mediante la lectura de una sola pieza; las ventajas que aportan las herramientas digitales, pese a ser notables, no alteran la esencia del método. Cambiemos el enfoque y pensemos en términos de lectura distante\index{lectura distante}. ¿Y si analizáramos la variación de las configuraciones escénicas de las comedias de Calderón según su género? La extracción de datos y el cálculo no diferiría en su grado de complejidad, pero ahora requeriría un tiempo notablemente mayor porque obligaría a repetir el mismo proceso para cada jornada de cada una de las piezas. ¿Y si queremos comparar autos sacramentales con comedias? ¿O el corpus de Calderón con el de Lope? No es necesario volver a decir que la posibilidad teórica de llevar algo a cabo, con harta frecuencia, se da de bruces en la práctica con los siempre escasos recursos humanos y temporales.

De acuerdo con \citeauthor{ilsemann1998}~\parencite*[15-16]{ilsemann1998}, nos encontramos con dos vías de investigación opuestas. En una dirección se hallaría la búsqueda heurística de diferencias en un corpus textual completo, como, por ejemplo, la caracterización tipológica de un género. Pensemos para ilustrarlo en las divergencias entre una comedia y una tragedia si consideramos parámetros tales como el número de personajes en escena, el tipo de personaje\index{personaje} o la longitud de los parlamentos\index{parlamento}. En el teatro del Siglo de Oro, además, podríamos sumar a las variables puramente teatrales otras métricas. Cabría especular, por poner el caso, que, si la interpolación de consonancias en tiradas asonantes aparece de manera habitual en los pasajes cómicos, una frecuencia baja de este recurso en la obra bien podría sugerir un género más serio.

De la misma manera, se establecen criterios diferenciadores dentro de una sola obra dramática. Podría suponerse que los personajes se caracterizan mediante las relaciones cuantitativas de \textit{dominancia escénica}\index{dominancia escénica} (\Cref{chap:ejdram}). \citeauthor{ilsemann1998} advierte que el peso de tales relaciones lo confiere la recepción subjetiva, por lo que el análisis cuantitativo, en este caso, serviría para objetivizar tal interpretación. Por este motivo, una formulación deficiente de las hipótesis del análisis por computador conduce con frecuencia a resultados triviales. Sería un pobre aporte, por ejemplo, contar los parlamentos de \textit{El burlador de Sevilla}, comparar los resultados de los personajes y concluir de la abrumadora dominancia de don Juan Tenorio que se trata de un personaje protagonista. De la misma manera, tampoco diría nada nuevo comprobar que la densidad escénica es superior en las comedias, donde la claridad de la trama y las tipologías reconocibles de personaje\index{personaje} desempeñan un papel primordial.

Resultaría tal vez más productivo, creemos, emplear estas medidas para comparar valores semejantes. Esto es, contrastar la dominancia escénica entre los protagonistas de dos obras, la del protagonista por género o la densidad escénica entre los corpus de comedias de varias plumas. La obviedad, en esos casos, se diluiría al pasar de la clasificación categórica mayor al sutil continuo de los detalles que, ahora sí, escapan fácilmente al escrutinio del lector corriente. La mente humana ha evolucionado para clasificar el mundo a su alrededor en grandes grupos, conforme a unos pocos patrones más o menos abstractos. El computador, por el contrario, es capaz de trabajar con infinidad de detalles concretos para llevar a cabo su clasificación; esto es el fundamento de, sin ir más lejos, los análisis estilométricos para la atribución de autoría. 

En cualquier caso, debemos proporcionar dos cosas: la segmentación de la obra y los personajes que intervienen en cada segmento. Para lograrlo, debemos seguir caminos diversos dependiendo del criterio tomado para hacer las divisiones. Como norma común, los límites de un acto son los de uno o dos de sus segmentos. Si atendemos a un criterio sonoro, debemos dividir el texto según los metros de los versos\index{verso} que lo componen. Si pretendemos dividir en escenas, dado que no suelen indicarse explícitamente como con los actos, debemos hacer el seguimiento de los personajes que hay sobre las tablas en cada momento. Para esto, nos valdremos de las acotaciones personales operativas\footnote{Para encuadrar algunas de estas acotaciones aquí según la clasificación de \citeauthor{garcia2020}~\parencite*[76-79]{garcia2020} asumiremos que \textit{estar presente} es una acción.}.  La escena VI\footnote{Según las divisiones de la edición de Ciriaco Morón \parencite{calderon_lavidaessuenno}.} de \textit{La vida es sueño}\nocite{calderon_lavidaessuenno} acaba de la manera que vemos en el ejemplo \ref{ex:salenentran}\footnote{Morón se guía aquí por la edición de Madrid \parencite{calderon_madrid}. En la de Zaragoza \parencite{calderon_zaragoza} se lee en su lugar: «\textit{Vanse todos y queda el} \textsc{Rey}. Sale[n] \textsc{Crotaldo} [\emph{sic}], \textsc{Clarín},  \textsc{Rosaura}, \textit{y gente}».}

\begin{exe}
	\ex\label{ex:salenentran}\textit{(Éntranse todos. Antes de que se entre el} \textsc{Rey} \textit{sale} \textsc{Clotaldo}, \textsc{Rosaura} \textit{y} \textsc{Clarín}\textit{, y detienen al} \textsc{Rey}.)
\end{exe}

Esto es, se vacía el escenario (\textit{éntranse todos}) y salen (\textit{sale}) Clotaldo, Rosaura y Clarín, que toman a Basilio con ellos cuando no ha desaparecido del todo, pero ha abandonado el proscenio (\textit{antes de que se entre}). Por lo tanto, el programa debe señalar un cambio de nueve —\nolinebreak si consideramos dos soldados junto a Astolfo, dos damas de Estrella y dos acompañantes en la comitiva real\nolinebreak— a cero personajes, y de cero a cuatro. A continuación, en presencia de Rosaura y Clarín, conversan Clotaldo y el rey y, tras el cuarto parlamento\index{parlamento} de este, aparece la acotación de (\ref{ex:vase}) al pie de su parlamento. Al estar en singular, habremos de entender que solo le afecta a él. Quedan así  Clotaldo, Rosaura y Clarín en solitario ante el espectador.


\begin{exe}
	\ex\label{ex:vase}\textsc{Basilio.}\hspace{2.2cm} No te aflijas Clotaldo,\\
	\strut\hspace{3cm}si otro día hubiera sido,\hspace{3cm}875\\
	\strut\hspace{3cm}[...]\\
	\strut\hspace{3cm}descuidos vuestros, perdono.\\
	\strut\hspace{5.2cm}\textit{(Vase.)}
\end{exe}

Insistimos: la división en escenas responde en muchos casos a una cuestión interpretativa, por lo que, al extraer los elementos del texto debemos presentar la evidencia sin hacer juicios valorativos, pero ofreciendo información objetiva que facilite hacerlos. Para nuestro trabajo, esto significa llevar la cuenta de personajes en escena y señalar cuando se ha producido un posible vaciado del escenario. La interpretación de estos datos queda al buen criterio del filólogo. Aceptará los vacíos como divisiones tal cual o, si es sensato, cotejará estas marcas con la información interna. La misma prevención concierne a la segmentación métrica. Aunque marquemos los cambios, únicamente la observación en detalle es suficiente para discernir los segmentos englobantes de los englobados.

No debemos quedarnos en los segmentos, por lo que profundizaremos un nivel más en la estructura dramática.  Ilsemann~\parencite*[11]{ilsemann1998} defiende que el parlamento\index{parlamento} es la unidad mínima natural del texto dramático apta para el análisis digital. Las unidades subordinadas no tienen sentido en este tipo de análisis. La palabra o el verso\index{verso} no constituyen unidades dramáticas por sí mismas. No quiere decir esto que no lo sean en otros ámbitos \textemdash hay estudios estilométricos que toman, por ejemplo, la palabra y el verso como unidades mínimas\index{estilometría} \parencite{kroll2022a}\textemdash{} ni que sean irrelevantes en el análisis dramático. Al contrario, estas subunidades caracterizan el parlamento, que definimos en función de sus elementos constitutivos.

Las unidades mínimas estructurales y sus componentes presentan una situación muy distinta a la de los segmentos. Al contrario que estos, parlamentos\index{parlamento}, líneas\index{línea}, versos\index{verso} y palabras están claramente definidos, por lo que establecer sus límites no es una cuestión interpretativa. En lo que a nuestro trabajo concierne, esto se traduce en identificar tales componentes de manera directa sin temor a incurrir en apreciaciones subjetivas que introduzcan un sesgo en los datos producidos.

Dicho de otro modo, para las escenas debemos limitarnos a presentar la información extraída, con el fin de facilitar el ulterior trabajo de segmentación; asimismo es posible presentar los segmentos métricos, pero la distinción entre macro\index{macrosecuencia} y microsecuencias\index{microsecuencia} requiere considerar otros factores que no se infieren por completo a partir de los elementos textuales. Los versos, líneas y parlamentos, por el contrario, son identificables y mensurables, por lo que, una vez procesado el texto, el investigador puede aplicar métodos estadísticos sin necesidad de un paso interpretativo intermedio. De esta forma, las medidas basadas en estos componentes son las que mejor se prestan a la lectura distante\index{lectura distante}, ya que permiten procesar un gran número de piezas teatrales y comparar valores dramétricos con una intervención humana mínima.

Ilsemann propone valorar los siguientes indicadores:
\begin{enumerate}
	\item El número de letras que constituyen el parlamento
	\item El número de palabras que constituyen el parlamento
	\item El número de parlamentos de un personaje
	\item El parlamento más largo del personaje
\end{enumerate}

Asimismo, propone considerar esos valores respecto al total del personaje y del texto. Esto, claro está, es el punto de partida, ya que los resultados de un personaje por sí mismo no dicen nada fuera del contexto que provee el resto de apreciaciones de conjunto, sea respecto a otro personaje, a un grupo de personajes o a la totalidad del elenco.

Tampoco debemos perder de vista que el teatro áureo está en verso\index{verso}. Esto aporta un componente adicional a la caracterización de los parlamentos\index{parlamento} que responde a otros criterios, como el número de versos, el número de sílabas o el tipo de metro empleado. Podemos incluso apelar a la prosodia y describir el parlamento según el número de acentos o los patrones rítmicos de sus versos.

Lo ideal sería, no obstante, poder combinar observaciones sobre los parlamentos\index{parlamento} con las configuraciones escénicas y otros elementos dramáticos para obtener valores tales como parlamentos ajenos en presencia o en ausencia del personaje, número de apartes en el parlamento\footnote{Cabría plantearse si han de distinguirse diferentes categorías de elementos en el texto dramático según si contribuyen al avance de la fábula o tienen carácter metadiscursivo. Volvemos aquí una vez más a los fundamentos que habían planteado Aristóteles o Hegel sobre la naturaleza de los géneros literarios y las características distintivas de cada uno.}, parlamentos ajenos desde la muerte del personaje, parlamento más largo de ese personaje, media de palabras por personaje o los porcentajes de parlamentos y palabras de un personaje respecto al total respectivo \parencite[12-13]{ilsemann1998}. Donde Ilsemann contempla palabras, nosotros también consideraremos otros elementos definitorios del parlamento, como, por ejemplo, los versos\index{verso}.

Cuantos más detalles del texto identifiquemos, obtengamos y dispongamos de la manera adecuada, más posibilidades hay de hacer uso de ellos para considerar la pieza teatral desde una perspectiva similar a la que hemos ofrecido u otra que explote los datos recopilados de manera diferente. 

Debemos desarrollar la segunda parte del trabajo con esto presente si queremos ser capaces de proporcionar resultados útiles más allá de la mera demostración teórica. Esto es, hemos de instruir al computador para componer cuadros de información aptos para ser interpretados digitalmente y realizar con ellos operaciones\footnote{Considérense operaciones similares a las presentadas en \Cref{chap:ejdram}, pero extrapolando a un corpus masivo de piezas teatrales 	lo hecho en ese ejemplo para una sola jornada.}. La elección de los elementos es vital porque no solo determinan la posibilidad de obtener resultados o contribuyen a optimizarlos, sino que los aspectos de la obra teatral que afloraren dependen de la elección de las partes observadas. Repetimos: aunque escape de nuestro propósito, nuestra labor es proporcionar suficientes elementos de juicio para llegar a un buen puerto, pero, incluso si proveemos los medios, de poco servirán si se carece del criterio filológico para emplearlos e interpretarlos con sensatez.
